\documentclass[11pt,a4paper]{article}
\usepackage[utf8]{inputenc}
\usepackage[czech]{babel}
\usepackage{amsmath}
\usepackage{amsfonts}
\usepackage{amssymb}
\usepackage{booktabs}
\usepackage{longtable}
\usepackage{graphicx}
\usepackage{float}
\usepackage{subcaption}
\usepackage[figurename=Obr., tablename=Tab.]{caption}
\usepackage{a4wide}
\usepackage{mhchem}
\usepackage{siunitx}
\usepackage{empheq}
\usepackage[unicode, pdfauthor={Michal Šesták}, pdftitle={<++>}, pdfsubject={<++>},colorlinks=true, linkcolor=red,
urlcolor=black, citecolor=blue]{hyperref}
\author{Michal Šesták}
\title{Určení přísunů radonu do zón proměřených objektů}
%\LTcapwidth=\textwidth
%\renewcommand{\arraystretch}{1.0}
\begin{document}
\maketitle
\tableofcontents
\section{Úvod}
Jako zóny byly brány jednotlivé podlaží zkoumaných objektů. Určení přísunů radonu do jednotlivých zón nebylo možné provést u objektů č. 5, 6 a 7, protože v nich byly naměřeny koncentrace radonu pouze v jedné zóně. Uvedené nejistoty jsou u všech veličin rovny směrodatným odchylkám průměrů, tj. je uvažován faktor pokrytí $k=1$.
\section{Objekt č. 1}
\begin{table}[H]
    \centering
    \caption{Průměrné koncentrace radonu a objemy všech místností ve všech podlažích.}
    \begin{tabular}{lll}
\toprule
podlazi & $OAR$ [\si{Bq/m^3}] & $V$ [\si{m^3}] \\
\midrule
0 &           1094$\pm$55 &         40$\pm$8 \\
1 &            562$\pm$20 &        84$\pm$10 \\
2 &              51$\pm$2 &        97$\pm$15 \\
\bottomrule
\end{tabular}

\end{table}
\begin{table}[H]
    \centering
    \caption{Průtoky vzduchu mezi podlažími v \si{m^3/hod}. Hodnota v $i$-tém řádku a $j$-tém sloupci představuje průtok vzduchu z $i$-tého podlaží do $j$-tého podlaží. Poslední sloupec představuje exfiltrace z jednotlivých podlaží do vnějšího prostředí a poslední řádek představuje infiltrace z vnějšího prostředí do jednotlivých zón.}
    \begin{tabular}{lr}
\midrule
$k_{12}$   &  $1,32 \pm0,37 $\\
$k_{13}$   &  $0,03 \pm0,03 $\\
$k_{21}$   &  $2,50 \pm0,75 $\\
$k_{23}$   &  $0,47 \pm0,16 $\\
$k_{31}$   &  $0,14 \pm0,12 $\\
$k_{32}$   &  $1,26 \pm0,37 $\\
&\\                       
$k_{1_E}$   & $45,26\pm 7,75$ \\
$k_{2_E}$   & $23,02\pm 4,17$ \\
$k_{3_E}$   & $16,24\pm 2,86$ \\
$k_{1_I}$   & $43,97\pm 7,80$ \\
$k_{2_I}$   & $23,40\pm 4,27$ \\
$k_{3_I}$   & $17,14\pm 2,89$ \\
\midrule
$n$         & $0,38 \pm 0,05$ \\
\bottomrule
\end{tabular}

\end{table}
\begin{table}[H]
    \centering
    \caption{Výsledné přísuny radonu pro několik případů koncentrací radonu ve vnějším prostředí. $Q_i$ značí přísun radonu do $i$-tého podlaží.}
    \begin{tabular}{llll}
\toprule
$OAR_{out}$ [\si{Bq/m^3}] & $Q_0$ $\left[\si{\frac{Bq}{m^3\cdot hod}}\right]$ & $Q_1$ $\left[\si{\frac{Bq}{m^3\cdot hod}}\right]$ & $Q_2$ $\left[\si{\frac{Bq}{m^3\cdot hod}}\right]$ \\
\midrule
0  &                                        1256$\pm$335 &                                          160$\pm$35 &                                             7$\pm$2 \\
5  &                                        1250$\pm$334 &                                          159$\pm$34 &                                             6$\pm$2 \\
10 &                                        1245$\pm$332 &                                          157$\pm$34 &                                             5$\pm$2 \\
20 &                                        1234$\pm$329 &                                          154$\pm$33 &                                             3$\pm$1 \\
30 &                                        1223$\pm$326 &                                          152$\pm$33 &                                             1$\pm$1 \\
\bottomrule
\end{tabular}

\end{table}

\section{Objekt č. 2}
\begin{table}[H]
    \centering
    \caption{Průměrné koncentrace radonu a objemy všech místností ve všech podlažích.}
    \begin{tabular}{lll}
\toprule
podlazi & $OAR$ [\si{Bq/m^3}] & $V$ [\si{m^3}] \\
\midrule
1 &           1357$\pm$41 &        91$\pm$11 \\
\bottomrule
\end{tabular}

\end{table}
\begin{table}[H]
    \centering
    \caption{Průtoky vzduchu mezi podlažími v \si{m^3/hod}. Hodnota v $i$-tém řádku a $j$-tém sloupci představuje průtok vzduchu z $i$-tého podlaží do $j$-tého podlaží. Poslední sloupec představuje exfiltrace z jednotlivých podlaží do vnějšího prostředí a poslední řádek představuje infiltrace z vnějšího prostředí do jednotlivých zón.}
    \begin{tabular}{lll}
\toprule
{} &       1 & vnější prostředí \\
\midrule
1                &       0 &           45$\pm$9 \\
vnější prostředí &  45$\pm$9 &                0 \\
\bottomrule
\end{tabular}

\end{table}
\begin{table}[H]
    \centering
    \caption{Výsledné přísuny radonu pro několik případů koncentrací radonu ve vnějším prostředí. $Q_i$ značí přísun radonu do $i$-tého podlaží.}
    \begin{tabular}{ll}
\toprule
$OAR_{out}$ [\si{Bq/m^3}] & $Q_1$ $\left[\si{\frac{Bq}{m^3\cdot hod}}\right]$ \\
\midrule
0  &                                         683$\pm$157 \\
5  &                                         681$\pm$157 \\
10 &                                         678$\pm$156 \\
20 &                                         673$\pm$155 \\
30 &                                         668$\pm$154 \\
\bottomrule
\end{tabular}

\end{table}

\section{Objekt č. 3}
\begin{table}[H]
    \centering
    \caption{Průměrné koncentrace radonu a objemy všech místností ve všech podlažích.}
    \begin{tabular}{lll}
\toprule
podlazi & $OAR$ [\si{Bq/m^3}] & $V$ [\si{m^3}] \\
\midrule
1 &          3042$\pm$108 &         77$\pm$8 \\
2 &             211$\pm$7 &         65$\pm$8 \\
\bottomrule
\end{tabular}

\end{table}
\begin{table}[H]
    \centering
    \caption{Průtoky vzduchu mezi podlažími v \si{m^3/hod}. Hodnota v $i$-tém řádku a $j$-tém sloupci představuje průtok vzduchu z $i$-tého podlaží do $j$-tého podlaží. Poslední sloupec představuje exfiltrace z jednotlivých podlaží do vnějšího prostředí a poslední řádek představuje infiltrace z vnějšího prostředí do jednotlivých zón.}
    \begin{tabular}{llll}
\toprule
{} &          1 &          2 & vnější prostředí \\
\midrule
1                &          0 &  3.2+/-0.9 &           50+/-9 \\
2                &  1.7+/-0.5 &          0 &          56+/-10 \\
vnější prostředí &     51+/-9 &    54+/-10 &                0 \\
\bottomrule
\end{tabular}

\end{table}
\begin{table}[H]
    \centering
    \caption{Výsledné přísuny radonu pro několik případů koncentrací radonu ve vnějším prostředí. $Q_i$ značí přísun radonu do $i$-tého podlaží.}
    \begin{tabular}{lll}
\toprule
$OAR_{out}$ [\si{Bq/m^3}] & $Q_1$ $\left[\si{\frac{Bq}{m^3\cdot hod}}\right]$ & $Q_2$ $\left[\si{\frac{Bq}{m^3\cdot hod}}\right]$ \\
\midrule
0  &                                        2120$\pm$418 &                                           37$\pm$55 \\
5  &                                        2117$\pm$417 &                                           32$\pm$55 \\
10 &                                        2113$\pm$416 &                                           28$\pm$54 \\
20 &                                        2107$\pm$415 &                                           20$\pm$53 \\
30 &                                        2100$\pm$414 &                                           11$\pm$52 \\
\bottomrule
\end{tabular}

\end{table}

\section{Objekt č. 4}
\begin{table}[H]
    \centering
    \caption{Průměrné koncentrace radonu a objemy všech místností ve všech podlažích.}
    \begin{tabular}{lll}
\toprule
podlazi & $OAR$ [\si{Bq/m^3}] & $V$ [\si{m^3}] \\
\midrule
1 &            433$\pm$22 &       119$\pm$20 \\
2 &             208$\pm$7 &       102$\pm$14 \\
\bottomrule
\end{tabular}

\end{table}
\begin{table}[H]
    \centering
    \caption{Průtoky vzduchu mezi podlažími v \si{m^3/hod}. Hodnota v $i$-tém řádku a $j$-tém sloupci představuje průtok vzduchu z $i$-tého podlaží do $j$-tého podlaží. Poslední sloupec představuje exfiltrace z jednotlivých podlaží do vnějšího prostředí a poslední řádek představuje infiltrace z vnějšího prostředí do jednotlivých zón.}
    \begin{tabular}{llll}
\toprule
{} &       1 &        2 & vnější prostředí \\
\midrule
1                &       0 &   15+/-4 &           27+/-7 \\
2                &  15+/-4 &        0 &           44+/-9 \\
vnější prostředí &  28+/-9 &  43+/-11 &                0 \\
\bottomrule
\end{tabular}

\end{table}
\begin{table}[H]
    \centering
    \caption{Výsledné přísuny radonu pro několik případů koncentrací radonu ve vnějším prostředí. $Q_i$ značí přísun radonu do $i$-tého podlaží.}
    \begin{tabular}{lll}
\toprule
$OAR_{out}$ [\si{Bq/m^3}] & $Q_1$ $\left[\si{\frac{Bq}{m^3\cdot hod}}\right]$ & $Q_2$ $\left[\si{\frac{Bq}{m^3\cdot hod}}\right]$ \\
\midrule
0  &                                          131+/-38 &                                           55+/-29 \\
5  &                                          130+/-37 &                                           53+/-29 \\
10 &                                          129+/-37 &                                           51+/-28 \\
20 &                                          126+/-36 &                                           46+/-27 \\
30 &                                          124+/-35 &                                           42+/-26 \\
\bottomrule
\end{tabular}

\end{table}

\section{Objekt č. 8}
\begin{table}[H]
    \centering
    \caption{Průměrné koncentrace radonu a objemy všech místností ve všech podlažích.}
    \begin{tabular}{lll}
\toprule
podlazi & $OAR$ [\si{Bq/m^3}] & $V$ [\si{m^3}] \\
\midrule
0 &              33$\pm$2 &        66$\pm$13 \\
1 &              61$\pm$3 &       105$\pm$11 \\
2 &              79$\pm$2 &       153$\pm$15 \\
\bottomrule
\end{tabular}

\end{table}
\begin{table}[H]
    \centering
    \caption{Průtoky vzduchu mezi podlažími v \si{m^3/hod}. Hodnota v $i$-tém řádku a $j$-tém sloupci představuje průtok vzduchu z $i$-tého podlaží do $j$-tého podlaží. Poslední sloupec představuje exfiltrace z jednotlivých podlaží do vnějšího prostředí a poslední řádek představuje infiltrace z vnějšího prostředí do jednotlivých zón.}
    \begin{tabular}{lllll}
\toprule
{} &        0 &                1 &                2 & vnější prostředí \\
\midrule
0                &        0 &        9.3+/-3.3 &            7+/-4 &          91+/-17 \\
1                &  41+/-12 &                0 &          30+/-14 &         135+/-33 \\
2                &  54+/-22 &          59+/-25 &                0 &  (5.3+/-1.3)e+02 \\
vnější prostředí &  11+/-31 &  (1.4+/-0.5)e+02 &  (6.1+/-1.3)e+02 &                0 \\
\bottomrule
\end{tabular}

\end{table}
\begin{table}[H]
    \centering
    \caption{Výsledné přísuny radonu pro několik případů koncentrací radonu ve vnějším prostředí. $Q_i$ značí přísun radonu do $i$-tého podlaží.}
    \begin{tabular}{llll}
\toprule
$OAR_{out}$ [\si{Bq/m^3}] & $Q_0$ $\left[\si{\frac{Bq}{m^3\cdot hod}}\right]$ & $Q_1$ $\left[\si{\frac{Bq}{m^3\cdot hod}}\right]$ & $Q_2$ $\left[\si{\frac{Bq}{m^3\cdot hod}}\right]$ \\
\midrule
0  &                                          -50+/-32 &                                           73+/-31 &                                          322+/-76 \\
5  &                                          -50+/-30 &                                           67+/-29 &                                          302+/-71 \\
10 &                                          -51+/-28 &                                           60+/-26 &                                          282+/-67 \\
20 &                                          -53+/-24 &                                           47+/-22 &                                          242+/-57 \\
30 &                                          -55+/-21 &                                           34+/-18 &                                          202+/-48 \\
\bottomrule
\end{tabular}

\end{table}
%\begin{thebibliography}{Mm99}
%\end{thebibliography}
%\pagestyle{empty}
%\section{Přílohy}
%\appendix
\end{document}
