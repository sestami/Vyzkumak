\documentclass{beamer}
\usefonttheme[onlymath]{serif}
\usepackage[orientation=portrait,size=a0,scale=1.4,debug]{beamerposter}
\mode<presentation>{\usetheme{ZH}}
\usepackage{chemformula}
\usepackage[utf8]{inputenc}
\usepackage[T1]{fontenc}
\usepackage[czech]{babel} % required for rendering German special characters
\usepackage{siunitx} %pretty measurement unit rendering
\usepackage{hyperref} %enable hyperlink for urls
\usepackage{ragged2e}
\usepackage[font=scriptsize,justification=justified, figurename=Obr., tablename=Tab.]{caption}
\usepackage{array,booktabs,tabularx}
\usepackage[version=4]{mhchem}
\usepackage{lmodern}
\usepackage{exscale}
\usepackage{siunitx}
\sisetup{output-decimal-marker = {,}}
\sisetup{separate-uncertainty}
\usepackage{collcell}
\usepackage{microtype}
\usepackage{multirow}
\usepackage{amsmath}
\usepackage{amsfonts}
\usepackage{amssymb}
\usepackage{subcaption}

\newcolumntype{Z}{>{\centering\arraybackslash}X} % centered tabularx columns
\sisetup{per=frac,fraction=sfrac}

\title{\huge Multi-kompártmentový přístup ke kvantifikaci  objemové rychlosti přísunu zdrojů radonu do budov s využitím měřené intenzity větrání pomocí techniky indikačních plynů}
\author[michal.sestak@suro.cz]{Michal Šesták$^{1,2}$, Karel Jílek$^{1}$}
%\institute[ETH]{$^{1}$Institute for Biomedical Engineering, ETH and University of Zurich \\ $^{2}$Preclinical Laboratory for Translational Research into Affective Disorders, DPPP, Psychiatric Hospital, University of Zurich}
\institute[SÚRO]{$^{1}$Státní ústav radiační ochrany, v. v. i. \\ $^{2}$Fakulta jaderná a fyzikálně inženýrská, ČVUT v Praze}
\date{5. listopadu 2019}

% edit this depending on how tall your header is. We should make this scaling automatic :-/
\newlength{\columnheight}
%\setlength{\columnheight}{104cm}
\setlength{\columnheight}{99cm}

\begin{document}
\shorthandoff{-}
\begin{frame}
\begin{columns}
	\begin{column}{.43\textwidth}
		\begin{beamercolorbox}[center]{postercolumn}
			\begin{minipage}{.98\textwidth}  % tweaks the width, makes a new \textwidth
				\parbox[t][\columnheight]{\textwidth}{ % must be some better way to set the the height, width and textwidth simultaneously
\begin{myblock}{Úvod}
Cílem této práce bylo odvodit matematický model, pomocí něhož by se daly lokalizovat a kvantifikovat zdroje radonu v budovách za využití techniky indikačních plynů, a následně jej ověřit na naměřených datech. Postup je následující: budova se rozdělí na několik kompártmentů (zón), v nichž předpokládáme homogenní koncentrace radonu, a poté v těchto kompártmentech provedeme simultánně měření intenzity větrání pomocí techniky indikačních plynů a měření OAR. Z naměřených veličin můžeme určit tzv. objemové rychlosti přísunů zdrojů radonu (zjednodušeně přísuny radonu), které kvantifikují množství radonu dostávajícího se do kompartmentů.

Pokud se proměřují vícepodlažní objekty, pak se za kompartmenty většinou berou jednotlivá patra. S touto metodou lze však měřit i jednopodlažních objektech nebo jenom v částech nějakého objektu, např. v bytech.

Model byl nejprve zevrubně ověřen na naměřených datech. Poté byla pro vyzkoušení a ověření celého procesu provedena tři měření (viz tab.~\ref{tab:prehled}) se známými přísuny radonu. Ty byly vytvořeny umístěním průtočných zdrojů radonu typu RF 2000 do objektů.
\begin{table}
    \scriptsize
    \centering
    \caption{Objekty, v nichž bylo provedeno měření přísunů radonu. $N$ je počet kompartmentů, na který byl daný objekt rozdělen.}
    \label{tab:prehled}
    	\begin{tabular}{lllll}
		\toprule
		Objekt & Rozsah měření & $t$ [dny] & Typ objektu &$N$\\
		\midrule
		Skála 75, okr. Havlíčkův Brod & 23. 5. -- 5. 6. 2019 & 14 & chata & 3\\
		Hálková 980, Humpolec & 5. 6. -- 20. 6. 2019 & 15 & byt & 4\\
		Anglická 574, Dobřichovice & 9. 7. -- 30. 7. 2019 & 22 & rodinný dům & 3\\
		\bottomrule
	\end{tabular}

\end{table}

Pro měření OAR byly použity TESLA TSR sondy a CANARY detektory.
%Dále bylo nutné znát objemy kompartmentů, k jejich změření byl použit laserový měřič vzdálenosti.

%- popsat ten princip celého toho měření a vyhodnocení (tj. i chromatograf) -> to teda až v tom dalším boxu

%Měření probíhala od května do července 2019 (přidej tu tabulku přehledovou o měřeních)
\end{myblock}\vfill


\begin{myblock}{Metoda indikačních plynů}
    Pomocí této metody můžeme určit objemové průtoky vzduchu mezi specifikovanými kompartmenty zkoumaného objektu, dále obj. průtoky vzduchu ze všech zón do vnějšího prostředí (exfiltrace). Z exfiltrací a objemů všech zón lze určit výměnu vzduchu objektu. Jedná se o pasivní techniku, a tudíž poskytuje pouze průměrné hodnoty za celou dobu měření. 

Principem techniky je detekce vhodně použitých indikačních plynů, jejichž zdroje (tzv. vyvíječe) jsou rozmístěny po objektu a které mají definovanou emisi do zón. Ze známosti těchto emisí do jednotlivých zón, z odezev detektorů a z dalších faktů je možné určit hledané obj. průtoky vzduchu.

Měření obvykle trvá 14 nebo 31 dní a je při něm nutno kontinuálně snímat teplotu v zónách. Dále musí platit $N_p\geq N$, kde $N_p$ je počet indikačních plynů a $N$ počet zón.

Vyhodnocení množství nasorbovaných plynů v detektorech se provádí pomocí plynového chromatografu s termální desorpcí.

Vše potřebné o této metodě je popsáno v metodice~\cite{metodika}.
\end{myblock}\vfill

\begin{myblock}{Určení přísunů radonu}

\begin{itemize}
    \item výpočet infiltrací:
        \begin{equation}
            k_{i_I}=k_{i_E}+\sum_{j=1}^{N} \left(k_{ij}-k_{ji}\right)
            \label{eq:infiltrace}
        \end{equation}
    \item soustava rovnic popisující systém:
        \begin{equation}
            0=\frac{1}{V_i}\left( \sum^{N+1}_{j=1}\overline{a_j} k_{ji}-\sum^{N+1}_{j=1}\overline{a_i} k_{ij}\right)-\lambda \overline{a_i} +\overline{Q_i}\,,\quad i\in \{1,2,\ldots,N\}\label{eq:odvozovani_rovnovaha}
        \end{equation}
\end{itemize}

\begin{table}[h]
\def\arraystretch{1.1}
\scriptsize
    \centering
    \caption{Značení a jednotky používaných veličin.}
    \label{tab:veliciny}
    \begin{tabular}{lp{.7\textwidth}l}
    %\begin{tabular}{lll}
        \toprule
        $N$   & počet kompartmentů/zón uvnitř zkoumaného objektu&[-]\\
        $V_i$ & objem $i$-té zóny& [\si{m^3}] \\
        $k_{ij}$ & objemový průtok vzduchu z $i$-té zóny do $j$-té zóny& [\si{m^3/hod}]\\
        $k_{i, N+1}$ & exfiltrace $i$-té zóny, ozn. $k_{i_E}$; index $N+1$ značí vnější prostředí & [\si{m^3/hod}]\\
        $k_{N+1, i}$ & infiltrace $i$-té zóny, ozn. $k_{i_I}$; index $N+1$ značí vnější prostředí &[\si{m^3/hod}]\\
        $a_i$ & OAR v $i$-té zóně& [\si{Bq/m^3}] \\
        $\lambda$ & přeměnová konstanta radonu& [\si{1/hod}]\\
        $Q_i$ & přísun radonu do $i$-té zóny& $\left[\si{\frac{Bq}{m^3\cdot hod}}\right]$ \\
        \bottomrule
    \end{tabular}
\end{table}
\end{myblock}\vfill

		}\end{minipage}\end{beamercolorbox}
	\end{column}



	\begin{column}{.57\textwidth}
		\begin{beamercolorbox}[center]{postercolumn}
			\begin{minipage}{.98\textwidth} % tweaks the width, makes a new \textwidth
				\parbox[t][\columnheight]{\textwidth}{ % must be some better way to set the the height, width and textwidth simultaneously


\begin{myblock}{Příklad měření a vyhodnocení}
    %\begin{figure}
        %\centering
        %\begin{subfigure}{0.47\textwidth}
            %\centering
            %\includegraphics[width=0.99\textwidth]{tabs/OAR_dohromady.png}
            %\caption{}
        %\end{subfigure}
        %\hspace{1em}
        %\begin{subfigure}{0.47\textwidth}
            %\centering
            %\includegraphics[width=0.99\textwidth]{tabs/OAR_CANARY.png}
            %\caption{alksdjf}
        %\end{subfigure}
    %\end{figure}
    \begin{columns}
        \centering
        \begin{column}{.69\textwidth}
            Jako příklad měření a vyhodnocení je uveden objekt Skála 75. Jedná se o chatu se sklepem, přízemím a prvním patrem. Do každého podlaží/zóny byly umístěny vyvíječe dvou typů tracerů, celkově tedy bylo použito šest typů tracerů, což při $N=3$ umožňuje vyhodnotit naměřená data vícero způsoby za použití různých kombinací indikačních plynů. Celkově bylo použito: 
            \begin{itemize}
                \item 14 vyvíječů, 12 TD detektorů, 3 teploměry
                \item 2 průtočné zdroje radonu (umístěny do sklepa a do kuchyně v přízemí)
                \item 4 TESLA TSR sondy, 4 CANARY detektory
            \end{itemize}
            V tab.~\ref{tab:skala75_OAR} jsou průměrné OAR naměřené TESLA TSR sondami a CANARY detektory, v tab.~\ref{tab:skala75_prutoky} jsou vypočítané obj. průtoky vzduchu za použití dvou kombinací tracerů. Zkratkami TMH, MCH, PCH a MDC, resp. TCE a PCE jsou označeny fluorované uhlovodíky, resp. chlorované uhlovodíky.
            %\begin{itemize}
                %\item TMH, MCH, PCH a MDC jsou fluorované uhlovodíky
                %\item TCE a PCE jsou chlorované uhlovodíky
            %\end{itemize}
V tab.~\ref{tab:skala75_Q} jsou určené přísuny radonu. Za povšimnutí stojí velká variabilita $Q_i$ v závislosti na použité kombinaci tracerů a OAR.
        \end{column}
        \begin{column}{.3\textwidth}

        \centering
            \begin{table}
                \scriptsize
                \centering
                \caption{Průměrné OAR naměřené TESLA TSR sondami a CANARY detektory.}
                \label{tab:skala75_OAR}
                \begin{tabular}{lrr}
\toprule
podlaží & TESLA TSR & CANARY \\
\midrule
sklep          & $458\pm33$ & $381\pm38$\\
přízemí kuchyň & $789\pm43$ & $419\pm42$\\
přízemí ložnice& $633\pm37$ & $465\pm47$\\
první patro    & $276\pm31$ & $156\pm16$\\
\bottomrule
\end{tabular}

            \end{table}

            \begin{table}
                \centering
                \scriptsize
                \caption{Obj. průtoky vzduchu a výměna vzduchu $n$ pro třetí a pátou kombinaci indikačních plynů (z celkového počtu osmi kombinací).}
                \label{tab:skala75_prutoky}
                \begin{tabular}{l>{\raggedleft\arraybackslash}p{4cm}>{\raggedleft\arraybackslash}p{4cm}}
\toprule
& (TMH, MCH, PCE)               & (TCE, MDC, PCE)\\
\midrule                                                
$k_{12}$ &  10,188$\pm$2,611    & 7,859$\pm$1,288\\
$k_{13}$ &   0,908$\pm$0,261    & 0,893$\pm$0,159\\
$k_{21}$ &   3,220$\pm$0,776    & 1,309$\pm$0,211\\
$k_{23}$ &   1,025$\pm$0,161    & 1,235$\pm$0,180\\
$k_{31}$ &  -0,061$\pm$0,016    &-0,025$\pm$0,005\\
$k_{32}$ &   0,774$\pm$0,117    & 0,922$\pm$0,136\\
&&\\                                           
$k_{1_E}$&  23,244$\pm$5,443    & 2,474$\pm$1,325\\
$k_{2_E}$&  36,712$\pm$4,240    &46,234$\pm$4,862\\
$k_{3_E}$&   7,850$\pm$0,853    & 7,670$\pm$0,848\\
$k_{1_I}$&  31,181$\pm$6,093    & 9,941$\pm$1,867\\
$k_{2_I}$&  29,994$\pm$5,043    &39,997$\pm$5,039\\
$k_{3_I}$&   6,630$\pm$0,914    & 6,439$\pm$0,891\\
\midrule                                         
$n$      &   0,287$\pm$0,036    & 0,239$\pm$0,028\\
\bottomrule
\end{tabular}

            \end{table}

        \end{column}
    \end{columns}
    %\centering
    %\begin{minipage}{.8\textwidth}
        \begin{table}
            \scriptsize
            \centering
            \caption{Vypočítané přísuny radonu při použití OAR z: (a) TESLA TSR sond, (b) CANARY detektorů. V posledním řádku jsou uvedeny známé přísuny radonu z průtočných zdrojů.}
            \label{tab:skala75_Q}
            \begin{subtable}{0.45\textwidth}
                \centering
                \caption{}
                \label{tab:skala75_Q_sondy}
                \input{tabs/standalone_Q_sondy.tex}
            \end{subtable}
            \begin{subtable}{0.45\textwidth}
                \centering
                \caption{}
                \label{tab:skala75_Q_CANARY}
                \input{tabs/standalone_Q_CANARY.tex}
            \end{subtable}
        \end{table}
    %\end{minipage}

\end{myblock}\vfill


\begin{myblock}{Výsledky ostatních měření}
    V tab.~\ref{tab:halkova980} a \ref{tab:anglicka574} jsou uvedeny vypočítané přísuny radonu do zbylých dvou proměřených objektů. ýměna vzduchu obj. Hálková 980 
TO DO: uvest vymeny vzduchu vsech mereni (v textu)
\begin{columns}
    \begin{column}{.5\textwidth}
        sdfkjaskdjflkajdf
    \end{column}
    \begin{column}{.5\textwidth}
        \begin{table}
            \centering
            \scriptsize
            \caption{Určené přísuny radonu do zón objektu Hálková 980.}
            \label{tab:halkova980}
            \begin{tabular}{lr@{${}\pm{}$}rr@{${}\pm{}$}rr@{${}\pm{}$}rr@{${}\pm{}$}r}
\toprule
   & \multicolumn{2}{r}{$Q_1$}  & \multicolumn{2}{r}{$Q_2$}  & \multicolumn{2}{r}{$Q_3$}  & \multicolumn{2}{r}{$Q_4$} \\
\midrule
(MDC, PCE, TCE, TMH) & 444&253 & $-$25&104 & 44&86 & -152&368 \\
(MDC, MCH, TCE, TMH) & 445&241 & $-$86&104 & 38&84 & -152&351 \\
\midrule
zdroje & 332&64 & 0&0 & 0&0 & 0&0 \\
\bottomrule
\end{tabular}

        \end{table}
        \begin{table}
            \centering
            \scriptsize
            \caption{Určené přísuny radonu do zón objektu Anglická 574.}
            \label{tab:anglicka574}
            \begin{tabular}{lr@{${}\pm{}$}rr@{${}\pm{}$}rr@{${}\pm{}$}r}
        \toprule
        &\multicolumn{2}{r}{$Q_1$} & \multicolumn{2}{r}{$Q_2$}  & \multicolumn{2}{r}{$Q_3$} \\
        \midrule
(MCH, MDC, PCH) & 1057&245 & $-$31&13 & 21&7\\
        \midrule
zdroje & 455&90 & 0&0 & 0&0\\
        \bottomrule
    \end{tabular}

        \end{table}
    \end{column}
\end{columns}
\end{myblock}\vfill

\begin{myblock}{Závěr}
    
\end{myblock}
\vfill

					\begin{myblock}{References}
						\footnotesize
                        \tiny
                        \begin{thebibliography}{Mn99}
                                \bibitem{japonci2} OKUYAMA, Hiroyasu, Yoshinori ONISHI, Shin-ichi TANABE a Seiichi KASHIHARA. Statistical data analysis method for multi-zonal airflow measurement using multiple kinds of perfluorocarbon tracer gas. Building and Environment [online]. 2009, 44(3), 546-557. DOI: 10.1016/j.buildenv.2008.04.014. ISSN 03601323. Dostupné z: \url{https://linkinghub.elsevier.com/retrieve/pii/S0360132308000905} 
                                \bibitem{sherman} SHERMAN, Max H., Iain S. WALKER a Melissa M. LUNDEN. Uncertainties in Air Exchange using Continuous-Injection, Long-Term Sampling Tracer-Gas Methods. International Journal of Ventilation [online]. 2016, 13(1), 13-28. DOI: 10.1080/14733315.2014.11684034. ISSN 1473-3315. Dostupné z: \url{http://www.tandfonline.com/doi/full/10.1080/14733315.2014.11684034}
                                \bibitem{metodika} JÍLEK, Karel; FROŇKA, Aleš. Metodika stanovení výměny vzduchu ve vnitřním ovzduší budov s využitím pasivních integrálních měřidel indikačních plynů (pro potřeby SÚJB) [online]. 2016 [cit. 2019-08-01]. Dostupné z: \url{https://www.sujb.cz/
fileadmin/sujb/docs/dokumenty/metodiky/Stanoveni_vymeny_vzduchu.pdf.}
                        \end{thebibliography}
					\end{myblock}\vfill
		}\end{minipage}\end{beamercolorbox}
	\end{column}
\end{columns}
\end{frame}
\end{document}
