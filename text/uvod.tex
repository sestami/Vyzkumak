\begin{table}[p]
\def\arraystretch{1.2}
    \centering
    \caption{Značení a jednotky používaných veličin.}
    \label{tab:veliciny}
    \begin{tabular}{lp{.7\textwidth}l}
    %\begin{tabular}{lll}
        \toprule
        $N$   & počet kompartmentů/zón uvnitř zkoumaného objektu&[-]\\
        $N_p$ & počet použitých indikačních plynů & [-]\\
        $R_{ki}$ & odezva TD detektorů na $k$-tý indikační plyn v $i$-té zóně &[\si{ng}]\\
        $U_k$ & odběrová rychlost TD detektorů $k$-tého indikačního plynu &$\left[\si{\frac{ng}{ppm\cdot min}}\right]$\\
        $M_{k}$ & molekulová hmotnost $k$-tého indikačního plynu &[\si{g/mol}]\\
        $T_i$ & průměrná teplota v $i$-té zóně v průběhu měření& [K] nebo [\si{\degree C}]\\
        $p_i$ & průměrný atmosférický tlak v $i$-té zóně v průběhu měření& [Pa]\\
        $V_{i}^{mol}$ & molární objem indikačních plynů při průměrné teplotě a tlaku v $i$-té zóně&[\si{dm^3/mol}]\\ 
        $dt$& doba měření ventilace objektu & [hod]\\
        $m_{ki}$ & emise $k$-tého indikačního plynu v $i$-té zóně & [\si{mg/hod}]\\
        $C_{ki}$ & hmotnostní koncentrace $k$-tého indikačního plynu v $i$-té zóně& [\si{mg/m^3}] \\
        $V_i$ & objem $i$-té zóny& [\si{m^3}] \\
        $k_{ij}$ & objemový průtok vzduchu z $i$-té zóny do $j$-té zóny& [\si{m^3/hod}]\\
        $k_{i, N+1}$ & exfiltrace $i$-té zóny, ozn. $k_{i_E}$; index $N+1$ značí vnější prostředí & [\si{m^3/hod}]\\
        $k_{N+1, i}$ & infiltrace $i$-té zóny, ozn. $k_{i_I}$; index $N+1$ značí vnější prostředí &[\si{m^3/hod}]\\
        $n$   & výměna vzduchu celého objektu & $[\si{1/hod}]$\\
        $a_i$ & OAR v $i$-té zóně& [\si{Bq/m^3}] \\
        $\dot{a_i}$ & časová derivace $a_i$ & $\left[\si{\frac{Bq}{m^3\cdot hod}}\right]$ \\
        $\lambda$ & přeměnová konstanta radonu& [\si{1/hod}]\\
        $Q_i$ & přísun radonu do $i$-té zóny& $\left[\si{\frac{Bq}{m^3\cdot hod}}\right]$ \\
        $W$ & radonová výdejnost radonového zdroje typu RF 2000 & [\si{Bq/hod}]\\
        $Q_{zdroj}$ & definovaný přísun radonu od zdroje typu RF 2000& $\left[\si{\frac{Bq}{m^3\cdot hod}}\right]$ \\
        \bottomrule
    \end{tabular}
\end{table}
\chapter{Úvod}
Tato práce pojednává o problematice zaměřené na určování zdrojů radonu v budovách. Vybral jsem si tuto problematiku, protože je blízká mému profesnímu zaměření a protože mi přijde různorodá. Při jejím zpracovávání bylo potřeba provádět měření, matematicky odvozovat, programovat a vyhodnocovat naměřená data. Navíc jsou mnou uvedené postupy využitelné při reálných měřeních, která se na SÚRO (Státní ústav radiační ochrany) provádějí rutinně, a proto má práce i skutečné uplatnění. 

Radon v budovách představuje vážné zdravotní riziko. V současné době existuje řada způsobů a doporučení měření a ochrany proti radonu, avšak někdy se může stát, že i přes veškerá opatření radon do budovy stále proniká. V tomto případě je důležité najít místo, kudy radon dovnitř proudí, což bývá často komplikované. Jednou z nejčastěji používaných metod za tímto účelem je tzv. Blower door test~\cite{wiki_blowerDoorTest}. Nevýhodou tohoto testu je jeho finanční náročnost, jeho provádění není snadné a je potřeba ho dělat za zvláštních podmínek. Metoda uvedená v této práci naopak nevyžaduje od obyvatel v podstatě žádné přizpůsobování, není tak nákladná a poskytuje další informace, které lze využít dalším způsobem. Při této metodě se budova
rozdělí na několik kompartmentů (nebo také zón), v nichž předpokládáme homogenní koncentraci radonu, a poté v těchto komparmentech provedeme simultánně měření intenzity větrání pomocí techniky indikačních plynů a měření koncentrace radonu. Z naměřených veličin můžeme určit tzv. objemové rychlosti přísunů zdrojů radonu, které kvantifikují množství radonu dostávajících se do kompartmentů. Díky tomu můžeme zúžit naše hledání zdrojů radonu v budově, což jsou převážně právě místa, kterými do se radon do budovy dostává.

Naneštěstí měření intenzity větrání pomocí techniky indikačních plynů je značně náchylná na dělání chyb. Člověk, který ho provádí, musí být s touto technikou dobře seznámen a měl by mít dobře naučené potřebné postupy. Dalším kamenem úrazu je věrohodnost naměřených koncentrací radonu, jelikož některé používané kontinuální monitory radonu neposkytují vždy spolehlivá data.

Cílem práce bylo vytvořit výpočetní model objemových rychlostí přísunů radonu do kompartmentů, na které je zkoumaná budova rozdělena, a následně tento model ověřit na naměřených datech. V dalších kapitolách se místo výrazu budova používá obecnější pojem objekt, můžeme totiž zkoumat i jednotlivé části budovy, např. byty.

