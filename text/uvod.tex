\begin{table}[p]
\def\arraystretch{1.2}
    \centering
    \caption{Značení a jednotky používaných veličin.}
    \label{tab:veliciny}
    \begin{tabular}{lp{.7\textwidth}l}
    %\begin{tabular}{lll}
        \toprule
        $N$   & počet kompartmentů/zón uvnitř zkoumaného objektu&[-]\\
        $N_p$ & počet použitých indikačních plynů & [-]\\
        $R_{ki}$ & odezva TD detektorů na $k$-tý indikační plyn v $i$-té zóně &[\si{ng}]\\
        $U_k$ & odběrová rychlost TD detektorů $k$-tého indikačního plynu &$\left[\si{\frac{ng}{ppm\cdot min}}\right]$\\
        $M_{k}$ & molekulová hmotnost $k$-tého indikačního plynu &[\si{g/mol}]\\
        $T_i$ & průměrná teplota v $i$-té zóně v průběhu měření& [K] nebo [\si{\degree C}]\\
        $p_i$ & průměrný atmosférický tlak v $i$-té zóně v průběhu měření& [Pa]\\
        $V_{i}^{mol}$ & molární objem indikačních plynů při průměrné teplotě a tlaku v $i$-té zóně&[\si{dm^3/mol}]\\ 
        $dt$& doba měření ventilace objektu & [hod]\\
        $m_{ki}$ & emise $k$-tého indikačního plynu v $i$-té zóně & [\si{mg/hod}]\\
        $C_{ki}$ & hmotnostní koncentrace $k$-tého indikačního plynu v $i$-té zóně& [\si{mg/m^3}] \\
        $V_i$ & objem $i$-té zóny& [\si{m^3}] \\
        $k_{ij}$ & objemový průtok vzduchu z $i$-té zóny do $j$-té zóny& [\si{m^3/hod}]\\
        $k_{i, N+1}$ & exfiltrace $i$-té zóny, ozn. $k_{i_E}$; index $N+1$ značí vnější prostředí & [\si{m^3/hod}]\\
        $k_{N+1, i}$ & infiltrace $i$-té zóny, ozn. $k_{i_I}$; index $N+1$ značí vnější prostředí &[\si{m^3/hod}]\\
        $n$   & výměna vzduchu celého objektu & $[\si{1/hod}]$\\
        $a_i$ & OAR v $i$-té zóně& [\si{Bq/m^3}] \\
        $\dot{a_i}$ & časová derivace $a_i$ & $\left[\si{\frac{Bq}{m^3\cdot hod}}\right]$ \\
        $\lambda$ & přeměnová konstanta radonu& [\si{1/hod}]\\
        $Q_i$ & přísun radonu do $i$-té zóny& $\left[\si{\frac{Bq}{m^3\cdot hod}}\right]$ \\
        \bottomrule
    \end{tabular}
\end{table}
\chapter{Úvod}
kompartment==zóna
