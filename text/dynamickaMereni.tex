\chapter{Dynamická měření}\label{navesti:dynamickaMereni}
Dynamická měření přísunů radonu jsem provedl u tří objektů, viz tab.~\ref{tab:dynMer_prehled}.
\begin{table}[ht]
	\centering
	\caption{Objekty, v nichž jsem provedl dynamická měření. $T$ značí dobu měření ve dnech (zaokrouhleno na celé dny včetně počátečního a posledního dne).}
	\label{tab:dynMer_prehled}
	\begin{tabular}{llll}
		\toprule
		Objekt & Rozsah měření & $T$ [dny] & Typ objektu\\
		\midrule
		Skála 75, okr. Havlíčkův Brod & 23. 5. -- 5. 6. 2019 & 14 & chata\\
		Hálková 980, Humpolec & 5. 6. -- 20. 6. 2019 & 15 & byt\\
		Anglická 574, Dobřichovice & 9. 7. -- 30. 7. 2019 & 22 & rodinný dům\\
		\bottomrule
	\end{tabular}
\end{table}

Vývoj OAR v čase v jednotlivých zónách byl měřen primárně TERA sondami~\cite{tera} a sekundárně měřiči radonu CANARY~\cite{canary}. CANARY měřáky byly použity jako záložní systém, tj. pokud by v některé zóně  TERA sonda selhala, pak by se OAR v této zóně brala z příslušného CANARY měřáku. %Zevrubné informace o těchto detektorech lze dohledat v kapitole~\ref{navesti:radon}.

Dále bylo potřeba měřit vývoj teploty, což je znalost nutná při vyhodnocování množství nasorbovaných indikačních plynů v TD detektorech. K tomuto účelu byly použity dataloggery teploty a vlhkosti testo 174H~\cite{testo}.

Před samotnými měřeními přísunů radonu v objektech bylo nejprve nutno provést srovnávací měření TERA sond, jelikož každá sonda má různou odezvu při stejné OAR. O tomto pojednává podkapitola~\ref{navesti:dynMer_TERA}. Další podkapitoly obsahují dynamická měření přísunů radonu v uvedených objektech. 
\section{TERA sondy}\label{navesti:dynMer_TERA}
Pro dynamická měření přísunů radonu mi byly poskytnuty čtyři TERA sondy s označením 8, 10, 88 a 112. Pro srovnání jejich odezev s reálnou hodnotou OAR byly vloženy do sudu (nádoba válcovitého tvaru) spolu s referenčním monitorem radonu AlphaGuard~\cite{alphaguard}. Hodnota OAR z AlphaGuardu byla brána jako reálná hodnota OAR. V obr.~\ref{fig:dynMer_sondySrovnani} jsou zobrazeny naměřené vývoje OAR v čase ze zkoumaných sond a z Alphaguardu, v tab.~\ref{tab:dynMer_sondy} jsou k vidění nejdůležitější statistiky naměřených dat z každého monitoru.

\begin{figure}[H]
	\centering
	\includegraphics[width=1\linewidth]{images/sondy_srovnani}
	\caption{Vývoj OAR naměřený zkoumanými sondami a referenčním AlphaGuardem.}
	\label{fig:dynMer_sondySrovnani}
\end{figure}
\begin{table}[ht]
	\centering
	\caption{Statistiky vývojů OAR naměřených TERA sondami a AlphaGuardem v \si{Bq/m^3}.}
	\label{tab:dynMer_sondy}
	\begin{tabular}{lrrrrrrr}
		\toprule
		ID sondy &  count &  mean    &  min &  25\% &  50\% &  75\% &  max \\
		\midrule
		8          &     71 &   369  &  252 &  337 &  368 &  405 &  488 \\
		10         &     71 &   228  &  121 &  198 &  232 &  256 &  308 \\
		88         &     71 &   198  &  140 &  170 &  198 &  220 &  273 \\
		112        &     71 &   354  &  251 &  318 &  349 &  399 &  484 \\
		\midrule
		AlphaGuard &     71 &   328  &  247 &  285 &  318 &  373 &  434 \\
		\bottomrule
	\end{tabular}
%&  std
      
%&   47
%&   41
%&   33
%&   58
      
%&   50
\end{table}

Pro opravu odezev byla zavedena pro každou sondu kalibrační konstanta $B$, která je definována následovně:
\begin{equation}
	B=\frac{OAR_A}{OAR_T}\,,
\end{equation}
kde $OAR_A$ je hodnota z AlphaGuardu a $OAR_T$ je hodnota z příslušné TERA sondy. Pro získání věrohodné hodnoty koncentrace radonu z naměřené hodnoty danou sondou pak stačí tuto naměřenou hodnotu přenásobit $B$ náležející této sondě.

Za $OAR_A$, resp. $OAR_T$ byly brány průměry naměřených OAR AlphaGuardem a danou Tera sondou. Relativní nejistoty kalibračních konstant byly odhadnuty na 10~\%. Určené kalibrační konstanty všech sond jsou k nahlédnutí na obr.~\ref{fig:dynMer_sondyB} a v tab.~\ref{tab:dynMer_sondyB}. %Nejistoty kalibračních konstant určeny nebyly vzhledem k tomu, že se jedná o "veličinu", která se může při různých vnějších podmínkách měnit. Hlavním ovlivňujícím faktorem je velikost aerosolů v měřeném prostředí. Tato nespolehlivost TERA sond v poskytování věrohodných dat je dalším důvodem, proč byly použity měřáky radonu CANARY. Takto můžeme srovnávat data z více zdrojů.
\begin{figure}[H]
	\centering
	\includegraphics[width=0.7\linewidth]{images/sondy_B}
	\caption{Kalibrační konstanty proměřených TERA sond. Černou čárou je vyznačen ideální případ, kdy se odezva sondy rovná skutečné OAR (resp. OAR naměřené AlphaGuardem).}
	\label{fig:dynMer_sondyB}
\end{figure}
\begin{table}[ht]
	\centering
	\caption{Kalibrační konstanty TERA sond odvozené od referenčního AlphaGuardu. Skutečná hodnota $OAR$ se vypočte ze vztahu $OAR=B\cdot OAR_T$, kde $OAR_T$ je naměřená obj. aktivita radonu danou TERA sondou. Nejistota kalibračních konstant byla odhadnuta na 10~\%.}
	\label{tab:dynMer_sondyB}
	\begin{tabular}{lr}
		\toprule
		ID sondy &     $B$ \\
		\midrule
		8   & 0,889+/-0,089\\
		10  & 1,440+/-0,140\\
		88  & 1,655+/-0,166\\
		112 & 0,925+/-0,093\\
		\bottomrule
	\end{tabular}
\end{table}
\section{Objekt Skála 75, okr. Havlíčkův Brod}
\subsection{Použitá měřidla}
\begin{itemize}
    \setlength\itemsep{0em}
	\item 14 vyvíječů (2x TMH, 2x TCE, 3x MDC, 3x MCH, 2x PCE, 2x PCH)
	\item 12 TD detektorů
	\item 4 CANARY monitory
	\item 4 TERA sondy
	\item 3 TESTO měřiče teploty a vlhkosti
	\item 2 zdroje radonu
\end{itemize}

\subsection{Naměřené OAR, objemy a teploty}

\begin{table}[H]
    \centering
    \caption{Objemy podlaží objektu, průměrné teploty naměřené v každém podlaží dataloggery testo 174H, odhadnuté atmosférické tlaky v každém podlaží a přiřazení číslování kompartmentů jednotlivým podlažím.}
    \label{tab:skala75_objemy}
    \begin{tabular}{lll}
\toprule
podlazi & $OAR$ [\si{Bq/m^3}] & $V$ [\si{m^3}] \\
\midrule
0 &              33+/-2 &        66+/-13 \\
1 &              61+/-3 &       105+/-11 \\
2 &              79+/-2 &       153+/-15 \\
\bottomrule
\end{tabular}

\end{table}
\begin{figure}[H]
    \centering
    \includegraphics[width=1\textwidth]{skala75/OAR_dohromady.png}
    \caption{Hodnoty OAR naměřené TERA sondami po aplikování kalibračních konstant (tab.~\ref{tab:dynMer_sondyB}). Pro další vyhodnocování byly OAR naměřené v přízemí v kuchyni a v ložnici zprůměrovány.}
    \label{fig:skala75_OARdohromady}
\end{figure}
\subsection{Objemové průtoky vzduchu}

\begin{table}[H]
    \centering
    \caption{Přehled použitých indikačních plynů. $M$ je molekulová hmotnost příslušného plynu, $U$ je jeho odběrová rychlost. Dále je uvedeno, v jakém podlaží byly vyvíječe plynů umístěny s jejich celkovými odpary za celou dobu měření. Význam označení podlaží je vysvětlen v tab. \ref{tab:rovMer_podlazi}.}
    \label{tab:skala75_indikacniPlyny}
    \begin{tabular}{lrr}
\toprule
plyn & zóna&odpar [\si{mg}]\\
\midrule
 MDC & 1&   1042           \\
 MCH &2&    989            \\
 PCE &2&    317            \\
 TCE &3&    841            \\
 TMH &4&    991            \\
\bottomrule
\end{tabular}

\end{table}
\begin{table}[H]
    \centering
    \caption{Odezvy TD detektorů $R$ na všechny použité indikační plyny ve všech zónách.}
    \label{tab:skala75_odezvyTD}
    \begin{tabular}{lrr}
\toprule
plyn & zóna  &    $R$\\
\midrule
MCH & 1 &    72$\pm$ 5 \\
    & 2 &  2375$\pm$99 \\
    & 3 &   203$\pm$ 9 \\
MDC & 1 &    69$\pm$ 2 \\
    & 2 &  1829$\pm$42 \\
    & 3 &   189$\pm$ 5 \\
PCE & 1 &     0$\pm$ 0 \\
    & 2 &    16$\pm$ 1 \\
    & 3 &   548$\pm$16 \\
PCH & 1 &    41$\pm$ 5 \\
    & 2 &   186$\pm$ 4 \\
    & 3 &   729$\pm$18 \\
TCE & 1 &   384$\pm$29 \\
    & 2 &   165$\pm$ 8 \\
    & 3 &   100$\pm$ 5 \\
TMH & 1 &   291$\pm$52 \\
    & 2 &   154$\pm$16 \\
    & 3 &    81$\pm$10 \\
\bottomrule
\end{tabular}

\end{table}

\begin{table}[H]
    \centering
    \caption{Objemové průtoky vzduchu v \si{m^3/hod} pro všechny kombinace aplikovaných indikačních plynů. $n$ je výměna vzduchu vypočtená ze vztahu~\eqref{eq:prutoky_n}, $[n]=\si{1/hod}$.}
    \label{tab:skala75Prutoky_celkove}
%\begin{tabular}{l>{\raggedleft\arraybackslash}p{2.5cm}>{\raggedleft\arraybackslash}p{2.5cm}>{\raggedleft\arraybackslash}p{2.5cm}>{\raggedleft\arraybackslash}p{2.5cm}}
%\toprule
%{} & (TMH, MDC, PCE) & (TMH, MDC, PCH) & (TMH, MCH, PCE) & (TMH, MCH, PCH)\\ 
%\midrule
%$k_{12}$ & 4.139$\pm$1.044 & 3.771$\pm$0.994 & 3.445$\pm$0.874 & 3.128$\pm$0.830 \\          
%$k_{13}$ & 0.478$\pm$0.133 & 1.892$\pm$0.533 & 0.495$\pm$0.136 & 1.956$\pm$0.541 \\          
%$k_{21}$ & 0.656$\pm$0.152 & 0.574$\pm$0.137 & 0.525$\pm$0.126 & 0.455$\pm$0.114 \\          
%$k_{23}$ & 0.202$\pm$0.030 & 0.800$\pm$0.119 & 0.167$\pm$0.026 & 0.660$\pm$0.104 \\          
%$k_{31}$ &-0.019$\pm$0.005 & 0.854$\pm$0.233 &-0.015$\pm$0.004 & 0.881$\pm$0.237 \\          
%$k_{32}$ & 0.231$\pm$0.034 & 1.882$\pm$0.282 & 0.192$\pm$0.029 & 1.561$\pm$0.240 \\          
%$k_{1_E}$&12.492$\pm$2.813 &11.640$\pm$2.705 &13.100$\pm$2.886 &12.163$\pm$2.768 \\          
%$k_{2_E}$& 7.169$\pm$0.789 & 6.809$\pm$0.790 & 5.990$\pm$0.690 & 5.674$\pm$0.686 \\          
%$k_{3_E}$& 1.930$\pm$0.212 & 5.749$\pm$0.804 & 1.964$\pm$0.213 & 6.010$\pm$0.804 \\          
%$k_{1_I}$&16.471$\pm$3.007 &15.875$\pm$2.943 &16.530$\pm$3.022 &15.912$\pm$2.952 \\          
%$k_{2_I}$& 3.657$\pm$1.318 & 2.530$\pm$1.313 & 3.045$\pm$1.122 & 2.099$\pm$1.114 \\          
%$k_{3_I}$& 1.462$\pm$0.255 & 5.793$\pm$1.039 & 1.478$\pm$0.256 & 5.836$\pm$1.032 \\          
%\midrule
%$n$      & 0.091$\pm$0.014 & 0.102$\pm$0.015 & 0.089$\pm$0.014 & 0.101$\pm$0.015 \\
%\bottomrule
%\end{tabular}
%\vspace{0.5cm}

%\begin{tabular}{l>{\raggedleft\arraybackslash}p{2.5cm}>{\raggedleft\arraybackslash}p{2.5cm}>{\raggedleft\arraybackslash}p{2.5cm}>{\raggedleft\arraybackslash}p{2.5cm}}
    %\toprule
    %{} & (TCE, MDC, PCE) & (TCE, MDC, PCH) & (TCE, MCH, PCE) & (TCE, MCH, PCH) \\
    %\midrule
%$k_{12}$ & 3.330$\pm$0.551 & 2.965$\pm$0.519 & 2.774$\pm$0.468 & 2.462$\pm$0.438 \\
%$k_{13}$ & 0.462$\pm$0.081 & 1.828$\pm$0.322 & 0.476$\pm$0.082 & 1.879$\pm$0.326 \\
%$k_{21}$ & 0.215$\pm$0.034 & 0.188$\pm$0.032 & 0.172$\pm$0.030 & 0.149$\pm$0.028 \\
%$k_{23}$ & 0.203$\pm$0.029 & 0.802$\pm$0.118 & 0.168$\pm$0.026 & 0.662$\pm$0.104 \\
%$k_{31}$ &-0.006$\pm$0.001 & 0.279$\pm$0.060 &-0.005$\pm$0.001 & 0.288$\pm$0.061 \\
%$k_{32}$ & 0.230$\pm$0.034 & 1.922$\pm$0.283 & 0.191$\pm$0.029 & 1.595$\pm$0.241 \\
%$k_{1_E}$& 1.805$\pm$0.627 & 0.865$\pm$0.619 & 2.329$\pm$0.606 & 1.303$\pm$0.596 \\
%$k_{2_E}$& 7.579$\pm$0.796 & 7.166$\pm$0.798 & 6.322$\pm$0.699 & 5.960$\pm$0.695 \\
%$k_{3_E}$& 1.918$\pm$0.212 & 6.281$\pm$0.810 & 1.954$\pm$0.213 & 6.565$\pm$0.810 \\
%$k_{1_I}$& 5.388$\pm$0.840 & 5.191$\pm$0.872 & 5.412$\pm$0.771 & 5.207$\pm$0.812 \\
%$k_{2_I}$& 4.436$\pm$0.970 & 3.269$\pm$1.000 & 3.696$\pm$0.843 & 2.714$\pm$0.863 \\
%$k_{3_I}$& 1.478$\pm$0.231 & 5.852$\pm$0.926 & 1.497$\pm$0.232 & 5.908$\pm$0.914 \\
%\midrule                                                                           
%$n$      & 0.048$\pm$0.006 & 0.061$\pm$0.007 & 0.045$\pm$0.005 & 0.059$\pm$0.007 \\
%\bottomrule
%\end{tabular}
\begin{tabular}{l>{\raggedleft\arraybackslash}p{2.5cm}>{\raggedleft\arraybackslash}p{2.5cm}>{\raggedleft\arraybackslash}p{2.5cm}>{\raggedleft\arraybackslash}p{2.5cm}}
\toprule
{} & (TMH, MDC, PCE) & (TMH, MDC, PCH) & (TMH, MCH, PCE) & (TMH, MCH, PCH)\\ 
\midrule
$k_{12}$ &12,262$\pm$3,129 &  11,759$\pm$3,078 &  10,188$\pm$2,611 &   9,746$\pm$2,563 \\          
$k_{13}$ & 0,855$\pm$0,255 &   3,372$\pm$1,013 &   0,908$\pm$0,261 &   3,573$\pm$1,036 \\          
$k_{21}$ & 4,028$\pm$0,940 &   3,507$\pm$0,847 &   3,220$\pm$0,776 &   2,780$\pm$0,700 \\          
$k_{23}$ & 1,240$\pm$0,183 &   4,889$\pm$0,724 &   1,025$\pm$0,161 &   4,031$\pm$0,635 \\          
$k_{31}$ &-0,076$\pm$0,020 &   3,524$\pm$0,958 &  -0,061$\pm$0,016 &   3,611$\pm$0,968 \\          
$k_{32}$ & 0,931$\pm$0,137 &   5,967$\pm$0,967 &   0,774$\pm$0,117 &   4,945$\pm$0,820 \\          
$k_{1_E}$&21,425$\pm$5,271 &  19,770$\pm$5,057 &  23,244$\pm$5,443 &  21,411$\pm$5,208 \\          
$k_{2_E}$&44,024$\pm$4,853 &  41,624$\pm$4,833 &  36,712$\pm$4,240 &  34,644$\pm$4,195 \\          
$k_{3_E}$& 7,712$\pm$0,849 &  24,294$\pm$3,199 &   7,850$\pm$0,853 &  25,127$\pm$3,209 \\          
$k_{1_I}$&30,590$\pm$6,206 &  27,869$\pm$6,140 &  31,181$\pm$6,093 &  28,339$\pm$6,017 \\          
$k_{2_I}$&36,099$\pm$5,855 &  32,294$\pm$5,917 &  29,994$\pm$5,043 &  26,764$\pm$5,073 \\          
$k_{3_I}$& 6,472$\pm$0,916 &  25,525$\pm$3,693 &   6,630$\pm$0,914 &  26,079$\pm$3,658 \\          
\midrule                                                                              
$n$      & 0,310$\pm$0,038 &   0,363$\pm$0,042 &   0,287$\pm$0,036 &   0,344$\pm$0,041 \\
\bottomrule
\end{tabular}
\vspace{0.5cm}

\begin{tabular}{l>{\raggedleft\arraybackslash}p{2.5cm}>{\raggedleft\arraybackslash}p{2.5cm}>{\raggedleft\arraybackslash}p{2.5cm}>{\raggedleft\arraybackslash}p{2.5cm}}
    \toprule
    {} & (TCE, MDC, PCE) & (TCE, MDC, PCH) & (TCE, MCH, PCE) & (TCE, MCH, PCH) \\
    \midrule
$k_{12}$ & 7,859$\pm$1,288 &   7,286$\pm$1,238 &   6,544$\pm$1,094 &   6,050$\pm$1,047 \\
$k_{13}$ & 0,893$\pm$0,159 &   3,523$\pm$0,631 &   0,927$\pm$0,162 &   3,647$\pm$0,641 \\
$k_{21}$ & 1,309$\pm$0,211 &   1,140$\pm$0,195 &   1,049$\pm$0,182 &   0,906$\pm$0,169 \\
$k_{23}$ & 1,235$\pm$0,180 &   4,874$\pm$0,715 &   1,023$\pm$0,159 &   4,025$\pm$0,628 \\
$k_{31}$ &-0,025$\pm$0,005 &   1,146$\pm$0,243 &  -0,020$\pm$0,004 &   1,176$\pm$0,245 \\
$k_{32}$ & 0,922$\pm$0,136 &   6,419$\pm$0,960 &   0,767$\pm$0,116 &   5,330$\pm$0,817 \\
$k_{1_E}$& 2,474$\pm$1,325 &   0,539$\pm$1,320 &   3,713$\pm$1,256 &   1,616$\pm$1,248 \\
$k_{2_E}$&46,234$\pm$4,862 &  43,556$\pm$4,848 &  38,543$\pm$4,268 &  36,229$\pm$4,226 \\
$k_{3_E}$& 7,670$\pm$0,848 &  26,236$\pm$3,237 &   7,815$\pm$0,852 &  27,185$\pm$3,242 \\
$k_{1_I}$& 9,941$\pm$1,867 &   9,061$\pm$1,942 &  10,155$\pm$1,683 &   9,231$\pm$1,776 \\
$k_{2_I}$&39,997$\pm$5,039 &  35,866$\pm$5,149 &  33,303$\pm$4,414 &  29,780$\pm$4,477 \\
$k_{3_I}$& 6,439$\pm$0,891 &  25,404$\pm$3,517 &   6,613$\pm$0,889 &  26,019$\pm$3,470 \\
\midrule                                                                                 
$n$      & 0,239$\pm$0,028 &   0,298$\pm$0,034 &   0,212$\pm$0,025 &   0,275$\pm$0,031 \\
\bottomrule
\end{tabular}
\end{table}

\subsection{Přísuny radonu}

\begin{table}[H]
    \centering
    \caption{Průměrné přísuny radonu souhrně pro všechny kombinace indikačních plynů pro dynamické vyhodnocení.}
    \label{tab:skala75_prisunyDynamicky}
   \begin{tabular}{llll}
\toprule
{} & $Q_0$ $\left[\si{\frac{Bq}{hod}}\right]$ & $Q_1$ $\left[\si{\frac{Bq}{hod}}\right]$ & $Q_2$ $\left[\si{\frac{Bq}{hod}}\right]$ \\
\midrule
(TMH, MDC, PCE) &                             13027+/-2859 &                             29843+/-3776 &                               1298+/-293 \\
(TMH, MDC, PCH) &                             12583+/-2800 &                             29221+/-3784 &                              4412+/-1171 \\
(TMH, MCH, PCE) &                             13503+/-2794 &                             24940+/-3272 &                               1424+/-288 \\
(TMH, MCH, PCH) &                             12995+/-2731 &                             24363+/-3263 &                              4899+/-1144 \\
(TCE, MDC, PCE) &                               4376+/-854 &                             31477+/-3494 &                               1284+/-278 \\
(TCE, MDC, PCH) &                               4233+/-882 &                             30807+/-3524 &                              4359+/-1091 \\
(TCE, MCH, PCE) &                               4539+/-770 &                             26351+/-3065 &                               1417+/-272 \\
(TCE, MCH, PCH) &                               4375+/-806 &                             25724+/-3070 &                              4872+/-1058 \\
\bottomrule
\end{tabular}

\end{table}
\subsubsection{Vyhodnocení v rovnovážném stavu}
\begin{table}[H]
    \centering
    \caption{Průměrné objemové koncentrace radonu naměřené TERA sondami umístěnými v uvedených podlažích. $\sigma_A$ je nejistota OAR typu A plynoucí ze statistického zpracování naměřených dat, $\sigma_B$ je nejistota OAR typu B plynoucí z nejistoty měřidla a $\sigma$ je kombinovaná nejistota OAR. Při určování přísunů radonu v rovnovážném stavu byla použita pouze nejistota typu B, tj. $\sigma_B$. V posledním sloupci je průměrná citlivost TERA sond vypočtená z naměřených dat (tj. z naměřeného počtu impulzů a naměřeného OAR). Tato citlivost byla použita pro výpočet $\sigma_B$.}
    \label{tab:skala75_OARprumerne}
    \begin{tabular}{llrrrrr}
\toprule
ID sondy&podlaží& OAR [\si{Bq/m^3}]& $\sigma_A$ & $\sigma_B$ &$\sigma$& prům. citlivost $\left[\si{\frac{imp}{hod}/\frac{Bq}{m^3}}\right]$\\ 
\midrule
8  &0 & 458 & 309 & 33 & 311&0,405\\
10 &1 & 789 & 485 & 43 & 487&0,433\\
112&1 & 633 & 282 & 37 & 284&0,464\\
88 &2 & 276 & 356 & 31 & 358&0,296\\
\bottomrule
    \end{tabular}
\end{table}

\begin{table}[H]
    \centering
    \caption{Přísuny radonu určené z průměrných hodnot OAR, tj. jako v rovnovážném měření.}
    \label{tab:skala75_prisunyRovnovazne}
   %\begin{tabular}{lrrS[table-format=2.0(2)]}
%\toprule
%použité tracery & $Q_0$ $\left[\si{\frac{Bq}{m^3\cdot hod}}\right]$ & $Q_1$ $\left[\si{\frac{Bq}{m^3\cdot hod}}\right]$ & {$Q_2$ $\left[\si{\frac{Bq}{m^3\cdot hod}}\right]$} \\
\begin{tabular}{lrr
        >{\collectcell\num}r<{\endcollectcell}
        @{${}\pm{}$}
        >{\collectcell\num}r<{\endcollectcell}
    }
\toprule
použité tracery & $Q_0$ & $Q_1$  & \multicolumn{2}{r}{$Q_2$} \\
\midrule
(TMH, MDC, PCE) & $335\pm90$ & $236\pm42$ & 18&6 \\
(TMH, MDC, PCH) & $323\pm88$ & $231\pm42$ & 63&24 \\
(TMH, MCH, PCE) & $347\pm89$ & $197\pm36$ & 19&6 \\
(TMH, MCH, PCH) & $334\pm87$ & $192\pm35$ & 70&24 \\
&&&\multicolumn{2}{r}{}\\
(TCE, MDC, PCE) & $111\pm28$ & $249\pm41$ & 17&6 \\
(TCE, MDC, PCH) & $108\pm28$ & $243\pm41$ & 62&23 \\
(TCE, MCH, PCE) & $115\pm26$ & $208\pm35$ & 19&6 \\
(TCE, MCH, PCH) & $111\pm27$ & $203\pm35$ & 70&23 \\
\bottomrule
\end{tabular}

\end{table}
%\subsection{Přeurčená varianta}

%\begin{table}[H]
    %\centering
    %\caption{Průtoky vzduchu mezi jednotlivými zónami. Vnější prostředí je označeno číslem 4.}
    %\label{tab:skala75Preurcena_prutoky}
    %\begin{tabular}{ll}
\toprule
k12 &  3.0+/-1.9 \\
k13 &  0.7+/-1.9 \\
k21 &  0.2+/-0.7 \\
k23 &  0.3+/-0.7 \\
k31 &  0.1+/-1.3 \\
k32 &  0.6+/-1.3 \\
k14 &  2.9+/-3.3 \\
k24 &  6.9+/-1.3 \\
k34 &  3.1+/-2.2 \\
k41 &      6+/-5 \\
k42 &  3.7+/-2.9 \\
k43 &      3+/-4 \\
\bottomrule
\end{tabular}

%\end{table}
%\begin{table}[H]
    %\centering
    %\caption{Statistiky vypočítaných přísunů radonu $Q$ do jednotlivých zón.}
    %\label{tab:skala75Preurcena_objemy}
    %\begin{tabular}{lrrr}
\toprule
{} &  $Q_0$ $\left[\si{\frac{Bq}{m^3\cdot hod}}\right]$ &  $Q_1$ $\left[\si{\frac{Bq}{m^3\cdot hod}}\right]$ &  $Q_2$ $\left[\si{\frac{Bq}{m^3\cdot hod}}\right]$ \\
\midrule
count &                                                309 &                                                309 &                                                309 \\
mean  &                                                 70 &                                                 34 &                                                 10 \\
std   &                                                134 &                                                156 &                                                101 \\
min   &                                               -991 &                                               -419 &                                               -305 \\
25\\%   &                                                 18 &                                                -14 &                                                -20 \\
50\\%   &                                                 52 &                                                 27 &                                                 -0 \\
75\\%   &                                                109 &                                                 86 &                                                 22 \\
max   &                                                879 &                                               2169 &                                                914 \\
\bottomrule
\end{tabular}

%\end{table}

%\begin{figure}[H]
    %\centering
    %\includegraphics[width=\textwidth]{skala75/prisuny_preurcena.png}
    %\caption{Přísuny radonu. Popisky v legendě značí podlaží.}
    %\label{fig:skala75Preurcena_prisuny}
%\end{figure}

\section{Objekt Hálková 980, Humpolec}
\subsection{Použitá měřidla}
\begin{itemize}
    \setlength\itemsep{0em}
	\item 20 vyvíječů (4x MDC, 4x MCH, 4x PCE, 4x TCE, 4x TMH)
	\item 8 TD detektorů
	\item 4 CANARY monitory
	\item 4 TERA sondy
	\item 3 TESTO měřiče teploty a vlhkosti
	\item 1 zdroj radonu
\end{itemize}

\subsection{Naměřené OAR, objemy a teploty}

\begin{table}[H]
    \centering
    \caption{Přiřazení číslování kompartmentů jednotlivým podlažím, objemy všech zón objektu, průměrné teploty naměřené v každém zóně TERA sondami, odhadnuté atmosférické tlaky v každém zóně a průměrné OAR naměřené TERA sondami v každé zóně.}
    \label{tab:halkova980_objemy}
    \begin{tabular}{lll}
\toprule
podlazi & $OAR$ [\si{Bq/m^3}] & $V$ [\si{m^3}] \\
\midrule
0 &              33+/-2 &        66+/-13 \\
1 &              61+/-3 &       105+/-11 \\
2 &              79+/-2 &       153+/-15 \\
\bottomrule
\end{tabular}

\end{table}
\begin{figure}[H]
    \centering
    \includegraphics[width=1\textwidth]{halkova980/OAR_dohromady.png}
    \caption{Hodnoty OAR naměřené TERA sondami po aplikování kalibračních konstant (tab.~\ref{tab:dynMer_sondyB}).}
    \label{fig:halkova980_OARdohromady}
\end{figure}

\subsection{Objemové průtoky vzduchu}

\begin{table}[H]
    \centering
    \caption{Přehled použitých indikačních plynů. $M$ je molekulová hmotnost příslušného plynu, $U$ je jeho odběrová rychlost. Dále je uvedeno, v jaké zóně byly vyvíječe plynů umístěny s jejich celkovými odpary za celou dobu měření. \ref{tab:rovMer_podlazi}.}
    \label{tab:halkova980_indikacniPlyny}
    \begin{tabular}{lrr}
\toprule
plyn & zóna&odpar [\si{mg}]\\
\midrule
 MDC & 1&   1042           \\
 MCH &2&    989            \\
 PCE &2&    317            \\
 TCE &3&    841            \\
 TMH &4&    991            \\
\bottomrule
\end{tabular}

\end{table}
\begin{table}[H]
    \centering
    \caption{Odezvy TD detektorů $R$ na všechny použité indikační plyny ve všech zónách.}
    \label{tab:halkova980_odezvyTD}
    \begin{tabular}{lrr}
\toprule
plyn & zóna  &    $R$\\
\midrule
MCH & 1 &    72$\pm$ 5 \\
    & 2 &  2375$\pm$99 \\
    & 3 &   203$\pm$ 9 \\
MDC & 1 &    69$\pm$ 2 \\
    & 2 &  1829$\pm$42 \\
    & 3 &   189$\pm$ 5 \\
PCE & 1 &     0$\pm$ 0 \\
    & 2 &    16$\pm$ 1 \\
    & 3 &   548$\pm$16 \\
PCH & 1 &    41$\pm$ 5 \\
    & 2 &   186$\pm$ 4 \\
    & 3 &   729$\pm$18 \\
TCE & 1 &   384$\pm$29 \\
    & 2 &   165$\pm$ 8 \\
    & 3 &   100$\pm$ 5 \\
TMH & 1 &   291$\pm$52 \\
    & 2 &   154$\pm$16 \\
    & 3 &    81$\pm$10 \\
\bottomrule
\end{tabular}

\end{table}

\begin{table}[H]
    \centering
    \caption{Objemové průtoky vzduchu mezi zónami v \si{m^3/hod} a výměna vzduchu $n$ v \si{hod^{-1}}.}
    \label{tab:halkova980_prutoky}
    \begin{tabular}{l
        >{\collectcell\num}r<{\endcollectcell}
        @{${}\pm{}$}
        >{\collectcell\num}r<{\endcollectcell}
        >{\collectcell\num}r<{\endcollectcell}
        @{${}\pm{}$}
        >{\collectcell\num}r<{\endcollectcell}
}
%\begin{tabular}{l>{\raggedleft\arraybackslash}p{2.5cm}>{\raggedleft\arraybackslash}p{2.5cm}}
\toprule
%{} & (MDC, PCE, TCE, TMH) & (MDC, MCH, TCE, TMH) \\
{} & \multicolumn{2}{r}{(MDC, PCE,} & \multicolumn{2}{r}{(MDC, MCH,} \\
{} & \multicolumn{2}{r}{TCE, TMH)} &   \multicolumn{2}{r}{TCE, TMH)} \\
\midrule
$k_{12}$          &    12,2&4,9 &          40,4&13,4   \\
$k_{13}$          &     4,8&8,9 &           11,9&8,2   \\
$k_{14}$          &   96,0&35,9 &          92,9&34,0   \\
$k_{21}$          &   31,3&27,5 &          40,5&14,8   \\
$k_{23}$          &    17,9&7,9 &            4,4&3,3   \\
$k_{24}$          &   13,8&24,1 &          19,7&12,7   \\
$k_{31}$          &   63,4&26,0 &          62,6&26,1   \\
$k_{32}$          &     1,9&2,8 &            6,2&8,9   \\
$k_{34}$          &   46,9&24,3 &          46,5&23,9   \\
$k_{41}$          &   76,3&37,1 &          74,6&37,7   \\
$k_{42}$          &     4,1&4,1 &          13,6&13,2   \\
$k_{43}$          &    18,0&9,7 &           20,4&9,9   \\
&\multicolumn{2}{r}{}&\multicolumn{2}{r}{}\\
$k_{1_E}$          &   72,6&18,6 &          45,3&11,8   \\
$k_{2_E}$          &  -39,9&15,3 &           12,1&4,6   \\
$k_{3_E}$          &  -27,3&12,3 &          -31,5&8,8   \\
$k_{4_E}$          &   50,9&18,7 &          41,7&13,4   \\
$k_{1_I}$          &   14,7&67,4 &          12,7&62,2   \\
$k_{2_I}$          &    5,0&41,0 &          16,6&29,1   \\
$k_{3_I}$          &   44,2&40,8 &          47,1&39,8   \\
$k_{4_I}$          &   -7,5&65,6 &          -8,8&61,4   \\
\midrule
$n$  &     0,4&0,2 &            0,4&0,1   \\   
\bottomrule
\end{tabular}

    
    
    
    
    
    
    
    
    
    
    
    
    
    
    
    
    
    
    
    
    


\end{table}

\subsection{Přísuny radonu}

%\begin{table}[H]
    %\centering
    %\caption{Statistiky vypočítaných přísunů radonu $Q$ do jednotlivých podlaží.}
    %\label{tab:halkova980_prisuny}
    %\begin{tabular}{lrrr}
\toprule
{} &  $Q_0$ $\left[\si{\frac{Bq}{m^3\cdot hod}}\right]$ &  $Q_1$ $\left[\si{\frac{Bq}{m^3\cdot hod}}\right]$ &  $Q_2$ $\left[\si{\frac{Bq}{m^3\cdot hod}}\right]$ \\
\midrule
count &  478 &  478 & 478 \\
mean  & 1041 &  -22 &  19 \\
%std  &  279 &   44 &  37 \\
min   & -161 & -188 & -60 \\
25\%  &  869 &  -51 &  -6 \\
50\%  & 1044 &  -23 &  13 \\
75\%  & 1233 &    6 &  41 \\
max   & 1662 &   93 & 153 \\
\bottomrule
\end{tabular}

%\end{table}
\begin{table}[H]
    \centering
    \caption{Průměrné přísuny radonu do zón (ne podlaží!) pro všechny možné kombinace indikačních plynů z rovnovážného vyhodnocení.}
    \label{tab:halkova980_prisunyRovnovazne}
    %\begin{tabular}{lrrS[table-format=2.0(2)]}
%\toprule
%použité tracery & $Q_0$ $\left[\si{\frac{Bq}{m^3\cdot hod}}\right]$ & $Q_1$ $\left[\si{\frac{Bq}{m^3\cdot hod}}\right]$ & {$Q_2$ $\left[\si{\frac{Bq}{m^3\cdot hod}}\right]$} \\
\begin{tabular}{lrr
        >{\collectcell\num}r<{\endcollectcell}
        @{${}\pm{}$}
        >{\collectcell\num}r<{\endcollectcell}
    }
\toprule
použité tracery & $Q_0$ & $Q_1$  & \multicolumn{2}{r}{$Q_2$} \\
\midrule
(TMH, MDC, PCE) & $335\pm90$ & $236\pm42$ & 18&6 \\
(TMH, MDC, PCH) & $323\pm88$ & $231\pm42$ & 63&24 \\
(TMH, MCH, PCE) & $347\pm89$ & $197\pm36$ & 19&6 \\
(TMH, MCH, PCH) & $334\pm87$ & $192\pm35$ & 70&24 \\
&&&\multicolumn{2}{r}{}\\
(TCE, MDC, PCE) & $111\pm28$ & $249\pm41$ & 17&6 \\
(TCE, MDC, PCH) & $108\pm28$ & $243\pm41$ & 62&23 \\
(TCE, MCH, PCE) & $115\pm26$ & $208\pm35$ & 19&6 \\
(TCE, MCH, PCH) & $111\pm27$ & $203\pm35$ & 70&23 \\
\bottomrule
\end{tabular}

\end{table}

\section{Objekt Anglická 574, Dobřichovice}

\subsection{Použitá měřidla}
\begin{itemize}
    \setlength\itemsep{0em}
	\item 12 vyvíječů (4x MCH, 4x MDC, 4x PCH)
	\item 12 TD detektorů
	\item 2 blank TD detektory 
	\item 4 CANARY monitory
	\item 4 TERA sondy
	\item 3 TESTO měřiče teploty a vlhkosti
	\item 2 zdroje radonu
\end{itemize}

\subsection{Naměřené OAR, objemy a teploty}

\begin{table}[H]
    \centering
    \caption{Objemy všech podlaží objektu, průměrné teploty naměřené v každém podlaží TERA sondami, odhadnuté atmosférické tlaky v každém podlaží, průměrné OAR naměřené TERA sondami v každé zóně a přiřazení číslování kompartmentů jednotlivým podlažím.}
    \label{tab:anglicka574_objemy}
    \begin{tabular}{lll}
\toprule
podlazi & $OAR$ [\si{Bq/m^3}] & $V$ [\si{m^3}] \\
\midrule
0 &              33+/-2 &        66+/-13 \\
1 &              61+/-3 &       105+/-11 \\
2 &              79+/-2 &       153+/-15 \\
\bottomrule
\end{tabular}

\end{table}
\begin{figure}[H]
    \centering
    \includegraphics[width=1\textwidth]{anglicka574/OAR_dohromady.png}
    \caption{Hodnoty OAR naměřené TERA sondami po aplikování kalibračních konstant (tab.~\ref{tab:dynMer_sondyB}).}
    \label{fig:anglicka574_OARdohromady}
\end{figure}

\subsection{Objemové průtoky vzduchu}

\begin{table}[H]
    \centering
    \caption{Přehled použitých indikačních plynů. $M$ je molekulová hmotnost příslušného plynu, $U$ je jeho odběrová rychlost. Dále je uvedeno, v jakém podlaží byly vyvíječe plynů umístěny s jejich celkovými odpary za celou dobu měření. Význam označení podlaží je vysvětlen v tab. \ref{tab:rovMer_podlazi}.}
    \label{tab:anglicka574_indikacniPlyny}
    \begin{tabular}{lrr}
\toprule
plyn & zóna&odpar [\si{mg}]\\
\midrule
 MDC & 1&   1042           \\
 MCH &2&    989            \\
 PCE &2&    317            \\
 TCE &3&    841            \\
 TMH &4&    991            \\
\bottomrule
\end{tabular}

\end{table}
\begin{table}[H]
    \centering
    \caption{Odezvy TD detektorů $R$ na všechny použité indikační plyny ve všech zónách.}
    \label{tab:anglicka574_odezvyTD}
    \begin{tabular}{lrr}
\toprule
plyn & zóna  &    $R$\\
\midrule
MCH & 1 &    72$\pm$ 5 \\
    & 2 &  2375$\pm$99 \\
    & 3 &   203$\pm$ 9 \\
MDC & 1 &    69$\pm$ 2 \\
    & 2 &  1829$\pm$42 \\
    & 3 &   189$\pm$ 5 \\
PCE & 1 &     0$\pm$ 0 \\
    & 2 &    16$\pm$ 1 \\
    & 3 &   548$\pm$16 \\
PCH & 1 &    41$\pm$ 5 \\
    & 2 &   186$\pm$ 4 \\
    & 3 &   729$\pm$18 \\
TCE & 1 &   384$\pm$29 \\
    & 2 &   165$\pm$ 8 \\
    & 3 &   100$\pm$ 5 \\
TMH & 1 &   291$\pm$52 \\
    & 2 &   154$\pm$16 \\
    & 3 &    81$\pm$10 \\
\bottomrule
\end{tabular}

\end{table}

\begin{table}[H]
    \centering
    \caption{Objemové průtoky vzduchu mezi zónami v \si{m^3/hod} a výměna vzduchu $n$ v \si{hod^{-1}}.}
    \label{tab:anglicka574_prutoky}
    \begin{tabular}{lr}
\toprule
$k_{12}$    &$2,32 \pm0,38$ \\
$k_{13}$    &$0,29 \pm0,09$ \\
$k_{21}$    &$2,96 \pm0,48$ \\
$k_{23}$    &$4,25 \pm0,65$ \\
$k_{31}$    &$0,18 \pm0,03$ \\
$k_{32}$    &$0,55 \pm0,08$ \\
&\\
$k_{1_E}$   &$22,05\pm2,53$ \\
$k_{2_E}$   &$25,24\pm2,80$ \\
$k_{3_E}$   &$30,74\pm3,07$ \\
$k_{1_I}$   &$21,52\pm2,60$ \\
$k_{2_I}$   &$29,59\pm2,94$ \\
$k_{3_I}$   &$26,93\pm3,14$ \\
\midrule
$n$ &    $0,29\pm0,04$ \\
\bottomrule
\end{tabular}

\end{table}
\subsection{Přísuny radonu}

\begin{figure}[ht]
    \begin{subfigure}{\textwidth}
        \centering
        \includegraphics[width=\textwidth]{anglicka574/prisuny.png}
        \caption{}
        \label{fig:anglicka574_prisuny}
    \end{subfigure}
    \begin{subfigure}{\textwidth}
        \centering
        \includegraphics[width=\textwidth]{anglicka574/prisuny_zoom.png}
        \caption{}
        \label{fig:anglicka574_prisunyZoom}
    \end{subfigure}
    \caption{V (a) jsou určené časové vývoje přísunů radonu do jednotlivých podlaží. V (b) jsou přiblížené přísuny radonu do přízemí a prvního patra. Oblasti označené zesvětlenou barvou značí nejistotu přísunů radonu při faktoru pokrytí $k=1$.}
\end{figure}
\begin{table}[H]
    \centering
    \caption{Statistiky vypočítaných přísunů radonu $Q$ do jednotlivých podlaží.}
    \label{tab:anglicka574_prisuny}
    \begin{tabular}{lrrr}
\toprule
{} &  $Q_0$ $\left[\si{\frac{Bq}{m^3\cdot hod}}\right]$ &  $Q_1$ $\left[\si{\frac{Bq}{m^3\cdot hod}}\right]$ &  $Q_2$ $\left[\si{\frac{Bq}{m^3\cdot hod}}\right]$ \\
\midrule
count &  478 &  478 & 478 \\
mean  & 1041 &  -22 &  19 \\
%std  &  279 &   44 &  37 \\
min   & -161 & -188 & -60 \\
25\%  &  869 &  -51 &  -6 \\
50\%  & 1044 &  -23 &  13 \\
75\%  & 1233 &    6 &  41 \\
max   & 1662 &   93 & 153 \\
\bottomrule
\end{tabular}

\end{table}
\begin{table}[H]
    \centering
    \caption{Průměrné přísuny radonu do všech podlaží z rovnovážného vyhodnocení.}
    \label{tab:anglicka574_prisunyRovnovazne}
    \begin{tabular}{rrr}
        \toprule
        $Q_0$ $\left[\si{\frac{Bq}{m^3\cdot hod}}\right]$& $Q_1$ $\left[\si{\frac{Bq}{m^3\cdot hod}}\right]$ & $Q_2$ $\left[\si{\frac{Bq}{m^3\cdot hod}}\right]$\\
        \midrule
        $1042\pm233$ & $-22\pm12$ & $19\pm9$\\
        \bottomrule
    \end{tabular}
\end{table}
