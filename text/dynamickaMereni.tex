\chapter{Dynamická měření}\label{navesti:dynamickaMereni}
Dynamická měření přísunů radonu jsem provedl u tří objektů, viz tab.~\ref{tab:dynMer_prehled}.
\begin{table}[ht]
	\centering
	\caption{Objekty, v nichž jsem provedl dynamická měření. $dt$ značí dobu měření ve dnech (zaokrouhleno na celé dny včetně počátečního a posledního dne). V posledním sloupci je počet zón, na které byl daný objekt rozdělen.}
	\label{tab:dynMer_prehled}
	\begin{tabular}{lllll}
		\toprule
		Objekt & Rozsah měření & $dt$ [dny] & Typ objektu &$N$\\
		\midrule
		Skála 75, okr. Havlíčkův Brod & 23. 5. -- 5. 6. 2019 & 14 & chata & 3\\
		Hálková 980, Humpolec & 5. 6. -- 20. 6. 2019 & 15 & byt & 4\\
		Anglická 574, Dobřichovice & 9. 7. -- 30. 7. 2019 & 22 & rodinný dům & 3\\
		\bottomrule
	\end{tabular}
\end{table}

Vývoj OAR v čase v jednotlivých zónách byl měřen primárně TERA sondami~\cite{tera} a sekundárně měřiči radonu CANARY~\cite{canary}. CANARY měřáky byly použity jako záložní systém, tj. pokud by v některé zóně  TERA sonda selhala, pak by se OAR v této zóně brala z příslušného CANARY měřáku. %Zevrubné informace o těchto detektorech lze dohledat v kapitole~\ref{navesti:radon}. 
Pokud bylo v jedné zóně použito více kontinuálních monitorů radonu, pak se z jejich dat udělal průměrný časový vývoj OAR v této zóně a s tím bylo dále počítáno.

Dále bylo potřeba měřit vývoj teploty, což je znalost nutná při vyhodnocování množství nasorbovaných indikačních plynů v TD detektorech a k určení hmotnostních koncentrací indikačních plynů  v zónách. K tomuto účelu byly použity dataloggery teploty a vlhkosti testo 174H~\cite{testo}. V případě objektu Hálková 980 byly k určení hmotnostních koncentrací použity teploty naměřené TERA sondami, jelikož v tomto objektu byl osazen pouze jeden datalogger testo 174H (jedná se jednopodlažní byt). 

Pro určení hmotnostních koncentrací indikačních plynů je dále nutno změřit průměrné atmosférické tlaky ve všech zónách. Jejich hodnoty však stačí znát pouze přibližně, protože moc neovlivňují výpočet koncentrací. Proto byly všechny potřebné atmosférické tlaky při vyhodnocování všech objektů brány rovny \SI{100}{kPa}.

Objemy všech objektů byly změřeny laserovým dálkoměrem BOSCH GLM 120 C~\cite{dalkomer}.

Do objektů byly při měřeních umístěny vždy jeden nebo dva zdroje radonu typu RF 2000 (viz podkapitola \ref{navesti:radon_zdroje}). U těchto zdrojů nás pro naše měření zajímají pouze radonové výdejnosti $W$, které představují definované absolutní přísuny radonu do zón, ve kterých jsou zdroje umístěny. Absolutními přísuny radonu je myšleno množství aktivity radonu, jenž se do zón dostane za hodinu, a tedy jejich jednotkou je \si{Bq/hod}. Radonové výdejnosti použitých zdrojů jsou uvedeny v tab.~\ref{tab:dynMer_zdroje}. Podělením $W$ objemem příslušné zóny můžeme dopočítat přísun radonu pocházející od daného zdroje do této zóny. 
\begin{table}[ht]
    \centering
    \caption{ID a radonové výdejnosti $W$ použitých radonových zdrojů typu RF 2000. Jako ID byla použita část výrobního čísla daného zdroje. Přebráno z certifikátů zdrojů, viz příloha~\ref{navesti:priloha_zdroje}.}
    \label{tab:dynMer_zdroje}
    \begin{tabular}{lr}
        \toprule
        ID & $W$ [\si{Bq/hod}]\\
        \midrule
        38 & 15588\\
        37 & 14436\\
        \bottomrule
    \end{tabular}
\end{table}

Před samotnými měřeními přísunů radonu v objektech bylo nejprve nutno provést srovnávací měření TERA sond, jelikož každá sonda má různou odezvu při stejné OAR. O tom pojednává podkapitola~\ref{navesti:dynMer_TERA}. Další podkapitoly obsahují dynamická měření přísunů radonu v uvedených objektech. Je potřeba dávat pozor na to, zdali je značení OAR a přísunů radonu bráno podle podlaží či podle zón. Pokud se jedná objekt, který nelze rozdělit na zóny podle podlaží, pak je značení bráno podle zón, v opačném případě je používáno značení podle podlaží.

\section{TERA sondy}\label{navesti:dynMer_TERA}

Pro dynamická měření přísunů radonu mi byly poskytnuty čtyři TERA sondy s označením 8, 10, 88 a 112. Pro srovnání jejich odezev s reálnou hodnotou OAR byly vloženy do sudu (nádoba válcovitého tvaru) spolu s referenčním monitorem radonu AlphaGuard (ozn. AG)~\cite{alphaguard}. Hodnota OAR z AG byla brána jako reálná hodnota OAR. V obr.~\ref{fig:dynMer_sondySrovnani} jsou zobrazeny naměřené vývoje OAR v čase ze zkoumaných sond a z AG, v tab.~\ref{tab:dynMer_sondy} jsou k vidění nejdůležitější statistiky naměřených dat z každého monitoru.

\begin{figure}[ht]
	\centering
	\includegraphics[width=1\linewidth]{images/sondy_srovnani}
	\caption{Vývoj OAR naměřený zkoumanými sondami a AG.}
	\label{fig:dynMer_sondySrovnani}
\end{figure}
\begin{table}[ht]
	\centering
	\caption{Statistiky vývojů OAR naměřených TERA sondami a AG v \si{Bq/m^3}.}
	\label{tab:dynMer_sondy}
	\begin{tabular}{lrrrrrrr}
		\toprule
		ID sondy &  count &  mean    &  min &  25\% &  50\% &  75\% &  max \\
		\midrule
		8          &     71 &   369  &  252 &  337 &  368 &  405 &  488 \\
		10         &     71 &   228  &  121 &  198 &  232 &  256 &  308 \\
		88         &     71 &   198  &  140 &  170 &  198 &  220 &  273 \\
		112        &     71 &   354  &  251 &  318 &  349 &  399 &  484 \\
		\midrule
		AG &     71 &   328  &  247 &  285 &  318 &  373 &  434 \\
		\bottomrule
	\end{tabular}
%&  std
      
%&   47
%&   41
%&   33
%&   58
      
%&   50
\end{table}

Pro opravu odezev byla zavedena pro každou sondu kalibrační konstanta $B$, která je definována následovně:
\begin{equation}
	B=\frac{OAR_A}{OAR_T}\,,
\end{equation}
kde $OAR_A$ je průměrná OAR z hodnot naměřených AG a $OAR_T$ je průměrná OAR z hodnot naměřených příslušnou TERA sondou. Pro získání věrohodné hodnoty koncentrace radonu z naměřené hodnoty danou sondou pak stačí tuto naměřenou hodnotu přenásobit $B$ náležející této sondě.

Relativní nejistoty kalibračních konstant byly odhadnuty na 10~\%. Určené kalibrační konstanty všech sond jsou k nahlédnutí v tab.~\ref{tab:dynMer_sondyB}. %Nejistoty kalibračních konstant určeny nebyly vzhledem k tomu, že se jedná o "veličinu", která se může při různých vnějších podmínkách měnit. Hlavním ovlivňujícím faktorem je velikost aerosolů v měřeném prostředí. Tato nespolehlivost TERA sond v poskytování věrohodných dat je dalším důvodem, proč byly použity měřáky radonu CANARY. Takto můžeme srovnávat data z více zdrojů.
%\begin{figure}[H]
	%\centering
	%\includegraphics[width=0.7\linewidth]{images/sondy_B}
	%\caption{Kalibrační konstanty proměřených TERA sond. Černou čárou je vyznačen ideální případ, kdy se odezva sondy rovná skutečné OAR (resp. OAR naměřené AlphaGuardem).}
	%\label{fig:dynMer_sondyB}
%\end{figure}
\begin{table}[ht]
	\centering
	\caption{Kalibrační konstanty TERA sond odvozené od referenčního AG. Skutečná hodnota $OAR$ se vypočte ze vztahu $OAR=B\cdot OAR_T$, kde $OAR_T$ je naměřená obj. aktivita radonu danou TERA sondou. Nejistota kalibračních konstant byla odhadnuta na 10~\%.}
	\label{tab:dynMer_sondyB}
	\begin{tabular}{lr}
		\toprule
		ID sondy &     $B$ \\
		\midrule
		8   & 0,889$\pm$0,089\\
		10  & 1,440$\pm$0,140\\
		88  & 1,655$\pm$0,166\\
		112 & 0,925$\pm$0,093\\
		\bottomrule
	\end{tabular}
\end{table}

\section{Objekt Skála 75, okr. Havlíčkův Brod}

Jedná se o chatu se sklepem, přízemím (které zahrnuje verandu) a prvním patrem. Rozdělení na kompartmenty bylo provedeno podle těchto podlaží. Do každého podlaží/zóny byly umístěny vyvíječe dvou typů indikačních plynů. Ve sklepě byly čtyři vyvíječe plynů TMH a TCE, v přízemí šest vyvíječů plynů MDC a MCH a v prvním patře čtyři vyvíječe plynů PCH a PCE. Dále byly umístěny TD detektory: dva do sklepa, šest do přízemí a čtyři do prvního patra. Do sklepa a prvního patra byly umístěny jeden CANARY měřák a jedna TERA sonda, v přízemí byly oba dva typy monitorů dvakrát. Nakonec byly dány do sklepa a přízemí průtočné zdroje radonu typu RF 2000 (do sklepa zdroj s označením 38, do přízemí s označením 37) a byly změřeny objemy. 

To, že jsme umístili do každé ze zón zdroje dvou různých indikačních plynů, umožnilo vyhodnocení měření ventilace vícero způsoby. Při vyhodnocování bylo totiž počítáno pouze s $N=N_p=3$, přičemž byly uvažovány plyny, jejichž zdroje byly v různých zónách (viz pravidla o osazování měřidel, podkapitola~\ref{navesti:prutoky_instalace}). To nám dává osm kombinací trojic použitých plynů a tedy osm různých způsobů vyhodnocení měření ventilaci a následně přísunů radonu.

V příloze~\ref{navesti:priloha_skala75} lze dohledat vyhodnocené veličiny z měření ventilace a OAR. Je tam také uveden výpočet nejistot průměrných hodnot OAR z dat naměřených TERA sondami (podle informací uvedených v podkapitole~\ref{navesti:radon_TERAnejistota}). 

V následujícím oddíle jsou uvedeny přesně definované přísuny radonu od zdrojů RF 2000 a dále průměrné přísuny radonu vypočítané z naměřených průtoků vzduchu mezi zónami a z OAR naměřených TERA sondami, resp. CANARY měřáky pro všechny kombinace indikačních plynů. V případě TERA sond proběhlo i dynamické vyhodnocení pro určení vývojů $Q_i(t)$, které lze vidět v příloze~\ref{navesti:priloha_skala75_prisuny}, v následujícím oddílu jsou pouze uvedeny zprůměrované hodnoty $Q_i(t)$.

V posledním oddílu této kapitoly je zobrazeno zpětné ověření OAR (pro všechny kombinace tracerů) a $k_{ij}$ (pro kombinaci (TMH, MCH, PCE)) ve smyslu podkapitoly~\ref{navesti:model_overeni}, navíc jsou tam uvedeny i průměrné OAR naměřené CANARY měřáky a $k_{ij}$ z vyhodnocení měření ventilace objektu při použití kombinace tracerů (TMH, MCH,
PCE). Zpětné ověření $k_{ij}$ bylo provedeno pouze pro tuto kombinaci tracerů.

%\subsection{Použitá měřidla}
%\begin{itemize}
    %\setlength\itemsep{0em}
	%\item 14 vyvíječů (2x TMH, 2x TCE, 3x MDC, 3x MCH, 2x PCE, 2x PCH)
	%\item 12 TD detektorů
	%\item 4 CANARY monitory
	%\item 4 TERA sondy
	%\item 3 TESTO měřiče teploty a vlhkosti
	%\item 2 zdroje radonu
%\end{itemize}

%\subsection{Naměřené OAR, objemy a teploty}

%\begin{table}[H]
    %\centering
    %\caption{Objemy podlaží objektu, průměrné teploty naměřené v každém podlaží dataloggery testo 174H, odhadnuté atmosférické tlaky v každém podlaží a přiřazení číslování kompartmentů jednotlivým podlažím. Význam označení podlaží je vysvětlen v tab. \ref{tab:rovMer_podlazi}.}
    %\label{tab:skala75_objemy}
    %\begin{tabular}{lll}
\toprule
podlazi & $OAR$ [\si{Bq/m^3}] & $V$ [\si{m^3}] \\
\midrule
0 &              33+/-2 &        66+/-13 \\
1 &              61+/-3 &       105+/-11 \\
2 &              79+/-2 &       153+/-15 \\
\bottomrule
\end{tabular}

%\end{table}
%\begin{figure}[H]
    %\centering
    %\includegraphics[width=1\textwidth]{skala75/OAR_dohromady.png}
    %\caption{Vývoj OAR naměřený TERA sondami po aplikování kalibračních konstant (tab.~\ref{tab:dynMer_sondyB}). Pro další vyhodnocování byly OAR naměřené v přízemí v kuchyni a v ložnici zprůměrovány.}
    %\label{fig:skala75_OAR_dohromady}
%\end{figure}
%\begin{figure}[H]
    %\centering
    %\includegraphics[width=1\textwidth]{skala75/OAR_CANARY.png}
    %\caption{Vývoj OAR naměřený CANARY měřáky.}
    %\label{fig:skala75_OAR_CANARY}
%\end{figure}
%\subsection{Objemové průtoky vzduchu}

%\begin{table}[H]
    %\centering
    %\caption{Přehled použitých indikačních plynů a umístění jejich vyvíječů v objektu. V posledním sloupci jsou celkové odpary plynů ze všech jim odpovídajících vyvíječů.}
    %\label{tab:skala75_indikacniPlyny}
    %\begin{tabular}{lrr}
\toprule
plyn & zóna&odpar [\si{mg}]\\
\midrule
 MDC & 1&   1042           \\
 MCH &2&    989            \\
 PCE &2&    317            \\
 TCE &3&    841            \\
 TMH &4&    991            \\
\bottomrule
\end{tabular}

%\end{table}
%\begin{table}[H]
    %\centering
    %\caption{Odezvy TD detektorů $R$ na všechny použité indikační plyny ve všech zónách.}
    %\label{tab:skala75_odezvyTD}
    %\begin{tabular}{lrr}
\toprule
plyn & zóna  &    $R$\\
\midrule
MCH & 1 &    72$\pm$ 5 \\
    & 2 &  2375$\pm$99 \\
    & 3 &   203$\pm$ 9 \\
MDC & 1 &    69$\pm$ 2 \\
    & 2 &  1829$\pm$42 \\
    & 3 &   189$\pm$ 5 \\
PCE & 1 &     0$\pm$ 0 \\
    & 2 &    16$\pm$ 1 \\
    & 3 &   548$\pm$16 \\
PCH & 1 &    41$\pm$ 5 \\
    & 2 &   186$\pm$ 4 \\
    & 3 &   729$\pm$18 \\
TCE & 1 &   384$\pm$29 \\
    & 2 &   165$\pm$ 8 \\
    & 3 &   100$\pm$ 5 \\
TMH & 1 &   291$\pm$52 \\
    & 2 &   154$\pm$16 \\
    & 3 &    81$\pm$10 \\
\bottomrule
\end{tabular}

%\end{table}

%\begin{table}[H]
    %\centering
    %\caption{Objemové průtoky vzduchu v \si{m^3/hod} pro všechny kombinace aplikovaných indikačních plynů. $n$ je výměna vzduchu vypočtená ze vztahu~\eqref{eq:prutoky_n}, $[n]=\si{1/hod}$.}
    %\label{tab:skala75_prutoky}
%\begin{tabular}{l>{\raggedleft\arraybackslash}p{2.5cm}>{\raggedleft\arraybackslash}p{2.5cm}>{\raggedleft\arraybackslash}p{2.5cm}>{\raggedleft\arraybackslash}p{2.5cm}}
%\toprule
%{} & (TMH, MDC, PCE) & (TMH, MDC, PCH) & (TMH, MCH, PCE) & (TMH, MCH, PCH)\\ 
%\midrule
%$k_{12}$ &12,262$\pm$3,129 &  11,759$\pm$3,078 &  10,188$\pm$2,611 &   9,746$\pm$2,563 \\          
%$k_{13}$ & 0,855$\pm$0,255 &   3,372$\pm$1,013 &   0,908$\pm$0,261 &   3,573$\pm$1,036 \\          
%$k_{21}$ & 4,028$\pm$0,940 &   3,507$\pm$0,847 &   3,220$\pm$0,776 &   2,780$\pm$0,700 \\          
%$k_{23}$ & 1,240$\pm$0,183 &   4,889$\pm$0,724 &   1,025$\pm$0,161 &   4,031$\pm$0,635 \\          
%$k_{31}$ &-0,076$\pm$0,020 &   3,524$\pm$0,958 &  -0,061$\pm$0,016 &   3,611$\pm$0,968 \\          
%$k_{32}$ & 0,931$\pm$0,137 &   5,967$\pm$0,967 &   0,774$\pm$0,117 &   4,945$\pm$0,820 \\          
%&&&&\\
%$k_{1_E}$&21,425$\pm$5,271 &  19,770$\pm$5,057 &  23,244$\pm$5,443 &  21,411$\pm$5,208 \\          
%$k_{2_E}$&44,024$\pm$4,853 &  41,624$\pm$4,833 &  36,712$\pm$4,240 &  34,644$\pm$4,195 \\          
%$k_{3_E}$& 7,712$\pm$0,849 &  24,294$\pm$3,199 &   7,850$\pm$0,853 &  25,127$\pm$3,209 \\          
%$k_{1_I}$&30,590$\pm$6,206 &  27,869$\pm$6,140 &  31,181$\pm$6,093 &  28,339$\pm$6,017 \\          
%$k_{2_I}$&36,099$\pm$5,855 &  32,294$\pm$5,917 &  29,994$\pm$5,043 &  26,764$\pm$5,073 \\          
%$k_{3_I}$& 6,472$\pm$0,916 &  25,525$\pm$3,693 &   6,630$\pm$0,914 &  26,079$\pm$3,658 \\          
%\midrule                                                                              
%$n$      & 0,310$\pm$0,038 &   0,363$\pm$0,042 &   0,287$\pm$0,036 &   0,344$\pm$0,041 \\
%\bottomrule
%\end{tabular}
%\vspace{0.5cm}

%\begin{tabular}{l>{\raggedleft\arraybackslash}p{2.5cm}>{\raggedleft\arraybackslash}p{2.5cm}>{\raggedleft\arraybackslash}p{2.5cm}>{\raggedleft\arraybackslash}p{2.5cm}}
    %\toprule
    %{} & (TCE, MDC, PCE) & (TCE, MDC, PCH) & (TCE, MCH, PCE) & (TCE, MCH, PCH) \\
    %\midrule
%$k_{12}$ & 7,859$\pm$1,288 &   7,286$\pm$1,238 &   6,544$\pm$1,094 &   6,050$\pm$1,047 \\
%$k_{13}$ & 0,893$\pm$0,159 &   3,523$\pm$0,631 &   0,927$\pm$0,162 &   3,647$\pm$0,641 \\
%$k_{21}$ & 1,309$\pm$0,211 &   1,140$\pm$0,195 &   1,049$\pm$0,182 &   0,906$\pm$0,169 \\
%$k_{23}$ & 1,235$\pm$0,180 &   4,874$\pm$0,715 &   1,023$\pm$0,159 &   4,025$\pm$0,628 \\
%$k_{31}$ &-0,025$\pm$0,005 &   1,146$\pm$0,243 &  -0,020$\pm$0,004 &   1,176$\pm$0,245 \\
%$k_{32}$ & 0,922$\pm$0,136 &   6,419$\pm$0,960 &   0,767$\pm$0,116 &   5,330$\pm$0,817 \\
%&&&&\\
%$k_{1_E}$& 2,474$\pm$1,325 &   0,539$\pm$1,320 &   3,713$\pm$1,256 &   1,616$\pm$1,248 \\
%$k_{2_E}$&46,234$\pm$4,862 &  43,556$\pm$4,848 &  38,543$\pm$4,268 &  36,229$\pm$4,226 \\
%$k_{3_E}$& 7,670$\pm$0,848 &  26,236$\pm$3,237 &   7,815$\pm$0,852 &  27,185$\pm$3,242 \\
%$k_{1_I}$& 9,941$\pm$1,867 &   9,061$\pm$1,942 &  10,155$\pm$1,683 &   9,231$\pm$1,776 \\
%$k_{2_I}$&39,997$\pm$5,039 &  35,866$\pm$5,149 &  33,303$\pm$4,414 &  29,780$\pm$4,477 \\
%$k_{3_I}$& 6,439$\pm$0,891 &  25,404$\pm$3,517 &   6,613$\pm$0,889 &  26,019$\pm$3,470 \\
%\midrule                                                                                 
%$n$      & 0,239$\pm$0,028 &   0,298$\pm$0,034 &   0,212$\pm$0,025 &   0,275$\pm$0,031 \\
%\bottomrule
%\end{tabular}
%\end{table}

\subsection{Přísuny radonu}
\begin{table}[H]
    \centering
    \caption{Přesně definované přísuny radonu ze zdrojů v \si{Bq/(m^3\cdot hod)}. Ve druhém sloupci je uvedeno, který zdroj byl umístěn v daném podlaží.}
    \label{tab:skala75_prisunyZdroj}
    \begin{tabular}{ll
        >{\collectcell\num}r<{\endcollectcell}
        @{${}\pm{}$}
        >{\collectcell\num}r<{\endcollectcell}}
        \toprule
        podlaží  &zdroj& \multicolumn{2}{r}{$Q_{zdroj}$}\\
        \midrule
        0 &38&400&51\\
        1 &37&114&13\\
        2 & NE &0&0\\
        \bottomrule
    \end{tabular}
\end{table}

\subsubsection{Vyhodnocení dynamického měření}
\begin{table}[H]
    \centering
    \caption{Průměrné přísuny radonu v \si{Bq/(m^3\cdot hod)} souhrnně pro všechny kombinace indikačních plynů určené zprůměrováním časových vývojů $Q_i(t)$ vypočítaných z dynamického vyhodnocení OAR naměřených TERA sondami. Závislosti $Q_i(t)$ lze vidět v příloze~\ref{navesti:priloha_skala75}.}
    \label{tab:skala75_prisunyDynamicky}
   \begin{tabular}{llll}
\toprule
{} & $Q_0$ $\left[\si{\frac{Bq}{hod}}\right]$ & $Q_1$ $\left[\si{\frac{Bq}{hod}}\right]$ & $Q_2$ $\left[\si{\frac{Bq}{hod}}\right]$ \\
\midrule
(TMH, MDC, PCE) &                             13027+/-2859 &                             29843+/-3776 &                               1298+/-293 \\
(TMH, MDC, PCH) &                             12583+/-2800 &                             29221+/-3784 &                              4412+/-1171 \\
(TMH, MCH, PCE) &                             13503+/-2794 &                             24940+/-3272 &                               1424+/-288 \\
(TMH, MCH, PCH) &                             12995+/-2731 &                             24363+/-3263 &                              4899+/-1144 \\
(TCE, MDC, PCE) &                               4376+/-854 &                             31477+/-3494 &                               1284+/-278 \\
(TCE, MDC, PCH) &                               4233+/-882 &                             30807+/-3524 &                              4359+/-1091 \\
(TCE, MCH, PCE) &                               4539+/-770 &                             26351+/-3065 &                               1417+/-272 \\
(TCE, MCH, PCH) &                               4375+/-806 &                             25724+/-3070 &                              4872+/-1058 \\
\bottomrule
\end{tabular}

\end{table}
\subsubsection{Vyhodnocení v rovnovážném stavu}
%\begin{table}[H]
    %\centering
    %\caption{Průměrné objemové koncentrace radonu naměřené TERA sondami umístěnými v uvedených podlažích. $\sigma_A$ je nejistota OAR typu A plynoucí ze statistického zpracování naměřených dat, $\sigma_B$ je nejistota OAR typu B plynoucí z ostatních zdrojů (statistika detekce, nejistota měřidla) a $\sigma$ je kombinovaná nejistota OAR. Při určování přísunů radonu v rovnovážném stavu byla použita pouze nejistota typu B, tj. $\sigma_B$. V posledním sloupci je průměrná citlivost TERA sond vypočtená z naměřených dat (tj. z naměřeného počtu impulzů a naměřeného OAR). Tato citlivost byla použita pro výpočet $\sigma_B$.}
    %\label{tab:skala75_OARprumerne}
    %\begin{tabular}{llrrrrr}
%\toprule
%ID sondy&podlaží& OAR [\si{Bq/m^3}]& $\sigma_A$ & $\sigma_B$ &$\sigma$& $\overline{c}$ $\left[\si{\frac{imp}{hod}/\frac{Bq}{m^3}}\right]$\\ 
%\midrule
%8  &0 & 458 & 309 & 33 & 311&0,405\\
%10 &1 & 789 & 485 & 43 & 487&0,433\\
%112&1 & 633 & 282 & 37 & 284&0,464\\
%88 &2 & 276 & 356 & 31 & 358&0,296\\
%\bottomrule
    %\end{tabular}
%\end{table}

%\begin{table}[H]
    %\centering
    %\caption{Průměrné OAR naměřené CANARY měřáky umístěnými v daných podlažích.}
    %\label{tab:skala75_OARprumerne_CANARY}
    %\begin{tabular}{llr}
        %\toprule
        %ID měřáku & podlaží & OAR [\si{Bq/m^3}]\\
        %\midrule
        %1 & 0 & $381\pm38$\\
        %2 & 1 & $419\pm42$\\
        %4 & 1 & $465\pm47$\\
        %3 & 2 & $156\pm16$\\
        %\bottomrule
    %\end{tabular}
%\end{table}

\begin{table}[ht]
    \centering
    \caption{Přísuny radonu určené z průměrných hodnot vývojů OAR naměřených TERA sondami. Jednotkou přísunů radonu je \si{Bq/(m^3\cdot hod)}.}
    \label{tab:skala75_prisunyRovnovazne}
   %\begin{tabular}{lrrS[table-format=2.0(2)]}
%\toprule
%použité tracery & $Q_0$ $\left[\si{\frac{Bq}{m^3\cdot hod}}\right]$ & $Q_1$ $\left[\si{\frac{Bq}{m^3\cdot hod}}\right]$ & {$Q_2$ $\left[\si{\frac{Bq}{m^3\cdot hod}}\right]$} \\
\begin{tabular}{lrr
        >{\collectcell\num}r<{\endcollectcell}
        @{${}\pm{}$}
        >{\collectcell\num}r<{\endcollectcell}
    }
\toprule
použité tracery & $Q_0$ & $Q_1$  & \multicolumn{2}{r}{$Q_2$} \\
\midrule
(TMH, MDC, PCE) & $335\pm90$ & $236\pm42$ & 18&6 \\
(TMH, MDC, PCH) & $323\pm88$ & $231\pm42$ & 63&24 \\
(TMH, MCH, PCE) & $347\pm89$ & $197\pm36$ & 19&6 \\
(TMH, MCH, PCH) & $334\pm87$ & $192\pm35$ & 70&24 \\
&&&\multicolumn{2}{r}{}\\
(TCE, MDC, PCE) & $111\pm28$ & $249\pm41$ & 17&6 \\
(TCE, MDC, PCH) & $108\pm28$ & $243\pm41$ & 62&23 \\
(TCE, MCH, PCE) & $115\pm26$ & $208\pm35$ & 19&6 \\
(TCE, MCH, PCH) & $111\pm27$ & $203\pm35$ & 70&23 \\
\bottomrule
\end{tabular}

\end{table}

\begin{table}[ht]
    \centering
    \caption{Přísuny radonu určené z průměrných hodnot vývojů OAR naměřených CANARY měřáky. Jednotkou přísunů radonu je \si{Bq/(m^3\cdot hod)}.}
    \label{tab:skala75_prisunyRovnovazneCANARY}
   \begin{tabular}{lllll}
\toprule
{} & $Q_1$ $\left[\si{\frac{Bq}{m^3\cdot hod}}\right]$ & $Q_2$ $\left[\si{\frac{Bq}{m^3\cdot hod}}\right]$ & $Q_3$ $\left[\si{\frac{Bq}{m^3\cdot hod}}\right]$ & $Q_4$ $\left[\si{\frac{Bq}{m^3\cdot hod}}\right]$ \\
\midrule
(MDC, PCE, TCE, TMH) &                                         446+/-260 &                                         -25+/-105 &                                           19+/-78 &                                        -150+/-382 \\
(MDC, MCH, TCE, TMH) &                                         455+/-250 &                                         -85+/-106 &                                            6+/-76 &                                        -140+/-365 \\
\bottomrule
\end{tabular}

\end{table}

\subsection{Zpětné ověření}

\subsubsection{OAR}
\begin{table}[H]
    \small
    \centering
    \caption{\small Průměrné OAR ve všech podlažích vypočítané pomocí rovnice~\eqref{eq:maticovy_zapis_rovnovaha} za použití průtoků vzduchu z tab.~\ref{tab:skala75_prutoky} a přísunů radonu pocházejících od zdrojů RF 2000 (tab.~\ref{tab:skala75_prisunyZdroj}) a průměrné OAR naměřené CANARY měřáky.}
    \label{tab:skala75_OAR_zpetne}
    \begin{tabular}{lrrr}
        \toprule
  použité tracery  & $a_1$ [\si{Bq/m^3}] &  $a_2$ [\si{Bq/m^3}]& $a_3$ [\si{Bq/m^3}] \\
        \midrule
(TMH, MDC, PCE) & $ 496\pm121$ & $410\pm72$ & $103\pm25$ \\
(TMH, MDC, PCH) & $ 496\pm119$ & $410\pm72$ & $107\pm26$ \\
(TMH, MCH, PCE) & $ 495\pm120$ & $467\pm83$ & $102\pm25$ \\
(TMH, MCH, PCH) & $ 495\pm118$ & $467\pm82$ & $107\pm26$ \\
(TCE, MDC, PCE) & $1417\pm333$ & $518\pm85$ & $209\pm50$ \\
(TCE, MDC, PCH) & $1416\pm338$ & $518\pm86$ & $219\pm50$ \\
(TCE, MCH, PCE) & $1415\pm319$ & $574\pm96$ & $209\pm49$ \\
(TCE, MCH, PCH) & $1415\pm324$ & $574\pm96$ & $218\pm49$ \\
\midrule
       naměřené & $381\pm\ \,38$ & $442\pm32$ & $156\pm16$ \\
        \bottomrule
    \end{tabular}
\end{table}

\subsubsection{Objemové průtoky vzduchu}

\begin{table}[H]
    \small
    \centering
    \caption{\small V prvních řádcích těchto tabulek označených \emph{zpětně} jsou průtoky vzduchu z dané zóny do ostatních zón a infiltrace této zóny vypočítané z rovnice~\eqref{eq:maticovy_zapis_rovnovaha} za využití znalosti ostatních průtoků vzduchu pro kombinaci indikačních plynů (TMH, MCH, PCE), viz tab.~\ref{tab:skala75_prutoky}, přísunů radonu pocházejících od zdrojů RF 2000 (tab.~\ref{tab:skala75_prisunyZdroj}) a průměrných OAR naměřených CANARY měřáky. V druhých řádcích tabulek označených \emph{měření} jsou pro srovnání uvedené příslušné průtoky vzduchu z tab.~\ref{tab:skala75_prutoky}. V (a) je zájmovou zónou první zóna, v (b) druhá zóna a v (c) třetí zóna.}
    \label{tab:skala75_prutoky_zpetne}
    \begin{subtable}{\textwidth}
        \centering
        \caption{}
        \begin{tabular}{lrrrrr}
\toprule
{} & $k_{12}$ [\si{m^3/hod}] & $k_{13}$ [\si{m^3/hod}] & $k_{14}$ [\si{m^3/hod}] & $k_{1_E}$ [\si{m^3/hod}] & $k_{1_I}$ [\si{m^3/hod}] \\
\midrule
zpětně &               $  24\pm14$ &                 $13\pm15$ &                $ 75\pm33$ &                 $51\pm11 $&                 $19\pm62 $\\
měření &               $  40\pm13$ &                 $ 12\pm\ \,8$ &                $ 93\pm34$ &                 $45\pm12 $&                 $13\pm62 $\\
\bottomrule
\end{tabular}

    \end{subtable}
    %\vspace{1em}

    \begin{subtable}{\textwidth}
        \centering
        \caption{}
        \begin{tabular}{lrrrrr}
\toprule
{} & $k_{21}$ [\si{m^3/hod}] & $k_{23}$ [\si{m^3/hod}] & $k_{24}$ [\si{m^3/hod}] & $k_{2E}$ [\si{m^3/hod}] & $k_{2I}$ [\si{m^3/hod}] \\
\midrule
zpětně &                99+/-112 &                  7+/-37 &               -20+/-102 &                 25+/-35 &                 29+/-45 \\
měření &                 40+/-15 &                   4+/-3 &                 20+/-13 &                  12+/-5 &                 17+/-29 \\
\bottomrule
\end{tabular}

    \end{subtable}
    %\vspace{1em}

    \begin{subtable}{\textwidth}
        \centering
        \caption{}
        \begin{tabular}{lrrrr}
\toprule
{} & $k_{31}$ [\si{m^3/hod}] & $k_{32}$ [\si{m^3/hod}] & $k_{3E}$ [\si{m^3/hod}] & $k_{3I}$ [\si{m^3/hod}] \\
\midrule
zpětně &          -24.45+/-17.34 &            1.34+/-16.37 &           27.71+/-20.50 &           26.49+/-20.50 \\
měření &            -0.06+/-0.02 &             0.77+/-0.12 &             7.85+/-0.85 &             6.63+/-0.91 \\
\bottomrule
\end{tabular}

    \end{subtable}
\end{table}

\subsection{Diskuze}
V diskuzi nejprve srovnáme průměrné přísuny radonu z dynamického a rovnovážného vyhodnocení  (tab.~\ref{tab:skala75_prisunyDynamicky}, tab.~\ref{tab:skala75_prisunyRovnovazne} a příloha~\ref{navesti:priloha_skala75_prisuny}). Bude následovat srovnání přísunů radonu vypočítaných při použití různých kombinací indikačních plynů při vyhodnocování objemových průtoků vzduchu (všechny tabulky uvádějící vypočítané přísuny radonu a příloha~\ref{navesti:priloha_skala75_prisuny}), dále srovnání přísunů určených z OAR naměřených TERA sondami a CANARY měřáky (tab.~\ref{tab:skala75_prisunyRovnovazne} a tab.~\ref{tab:skala75_prisunyRovnovazneCANARY}) a nakonec srovnání vypočítaných přísunů radonu s $Q_{zdroj}$ (tab.~\ref{tab:skala75_prisunyZdroj}). 

Také budeme diskutovat zpětné ověření OAR a objemových průtoků vzduchu z $Q_{zdroj}$ (tabulky~\ref{tab:skala75_OAR_zpetne} a \ref{tab:skala75_prutoky_zpetne}).

\subsubsection{Srovnání dynamického a rovnovážného vyhodnocení}
Z uvedených tabulek je zřejmé, že zprůměrováním vývojů $Q_i(t)$ jsme dostali v podstatě stejné hodnoty jako při rovnovážném vyhodnocení, což bylo očekáváno. Neshodování se hodnot by ukazovalo na nějakou chybu v kódu některého z vyhodnocovacích skriptů.

Výhodou dynamického vyhodnocení je, že nám dává závislosti přísunů radonu na čase, $Q_i(t)$. Můžeme tedy určit přísun radonu do jakékoliv zóny v jakýkoliv čas, který spadá do doby našeho měření ventilace a OAR. Problémem je, že jedinými veličinami, jejichž vývoj byl v čase změřen, jsou OAR v zónách. Pro objemové průtoky vzduchu mezi zónami máme stále pouze průměrné hodnoty, což znepřesňuje $Q_i(t)$ do takové míry, že je nelze rozumně použít. V příloze~\ref{navesti:priloha_skala75_prisuny} jsou $Q_i(t)$ uvedeny graficky i tabelovaně. Pro tento objekt se může zdát, že $Q_i(t)$ vycházejí celkem pěkně, avšak vzhledem k řečenému nemůžeme říct, jestli jsou to relevantní výsledky, či zdali se jedná o matoucí hodnoty. První by platilo, pokud by se $k_{ij}$
v průběhu času moc neměnily. To ale není moc pravděpodobné, protože průtoky vzduchu mezi zónami jsou značně ovlivněny venkovní teplotou, a proto zcela určitě $k_{ij}$ nabývá jiných hodnot např. ve sluneční den a v noci. Další faktory ovlivňující hodnoty $k_{ij}$ by bylo potřeba prozkoumat.

\subsubsection{Srovnání různých kombinací indikačních plynů}
Zde jsou rozdíly mezi jednotlivými přísuny radonu obrovské. Jsou způsobeny rozdílnými vyhodnocenými $k_{ij}$ pro různé kombinace použitých indikačních plynů (tab.~\ref{tab:skala75_prutoky}). Z uvedených hodnot přísunů radonu je zřetelné, že záměna jednoho plynu v některé ze zón vede hlavně ke změně přísunu radonu do této zóny, přísuny radonu do ostatních zón se změní vzhledem k jejich nejistotě víceméně zanedbatelně. Výjimku tvoří záměna plynů TMH a TCE v první zóně, které velmi změní přísun radonu do první zóny a trochu i do druhé zóny. U ostatních zón nevede záměna jejich zdrojového plynu k tak markantní změně přísunu radonu do nich jako v případě první zóny. 

Názorné zobrazení rozdílů při použití různých kombinací tracerů je možné vidět v grafech přílohy~\ref{navesti:priloha_skala75_prisuny}, ve kterých jsou zobrazeny závislosti $Q_i(t)$ z dynamického vyhodnocení.

Rozdíly mezi $k_{ij}$ určenými různými kombinacemi tracerů mohou být způsobeny rozdílným chováním použitých indikačních plynů při homogenizaci a proudění v objektu, dále chybným odečtením odparů některých vyvíječů, špatným vyhodnocením TD detektorů atp.

Díky znalosti $Q_{zdroj}$ a také díky v podstatě nulových přirozených přísunů radonu do objektu můžeme určit, která kombinace tracerů vede k nejlepším výsledkům. Jedná se o kombinaci (TMH, MCH, PCH). Plyny jsou v závorce uvedeny postupně pro první, druhou a třetí zónu. Při zpětném ověřování průtoků vzduchu z $Q_{zdroj}$ se proto uvažovala tato kombinace tracerů.

\subsubsection{Srovnání použití OAR z TERA sond a z CANARY měřáků}
Průměrné přísuny radonu vypočítané z OAR naměřené CANARY měřáky jsou menší než průměrné přísuny radonu vypočítané při použití OAR z TERA sond. Je to dáno tím, že průměrné OAR určené z dat naměřených CANARY měřáky jsou značně nižší než průměrné OAR určené z dat naměřených TERA sondami. Vzhledem ke známé nespolehlivosti a nepřesnosti TERA sond (viz určování kalibračních konstant v podkapitole~\ref{navesti:dynMer_TERA}) jsou považovány výpočty z dat naměřených CANARY měřáky za přesnější. CANARY měřáky jsou navíc odzkoušeny mnoha lety měření a projevily se jako spolehlivé monitory radonu. Je zajímavé, že i vývoje OAR se v případě CANARY měřáků a TERA sond liší, viz. obr.~\ref{fig:skala75_OAR_dohromady} a obr.~\ref{fig:skala75_OAR_CANARY}. Například u CANARY měřáků dosahuje OAR ve sklepě od 3. června do konce měření nejvyšších hodnot ze všech zón (až cca \SI{1400}{Bq/m^3}), zatímco v případě TERA sond je v
tomto období nejvyšší koncentrace v přízemní kuchyni (až cca \SI{1750}{Bq/m^3}). Jak již vyplývá z nutnosti zavádět kalibrační konstanty, problémem zde je rozdílná odezva jednotlivých TERA sond na stejnou hodnotu OAR. Dále lze z obrázků vypozorovat, že CANARY měřák v prvním patře zaznamenával většinu času nulovou hodnotu OAR, zatímco příslušná TERA sonda nulovou hodnotu v podstatě nikdy nenaměřila a navíc má zašuměnější data. Pokud budeme datům z CANARY měřáku věřit, pak to značí, že TERA sondy mají nenulou odezvu na nulovou OAR. 

V důsledku uvedených faktů se při zpětném ověřování počítalo s daty z CANARY měřáků.

\subsubsection{Srovnání vypočítaných přísunů radonu s $Q_{zdroj}$}
Pomocí tohoto srovnání byla vybrána kombinace tracerů (TMH, MCH, PCH) za davájící nejpřesnější výsledky. 

Dále lze vidět, že kombinace s TMH plynem v první zóně dávají v rámci nejistot správné hodnoty a že kombinace s TCE plynem v první zóně je nedávají.

\subsubsection{Zpětné ověření}
Zpětně dopočítané $k_{ij}$ pro kombinaci plynů (TMH, MCH, PCE) a $k_{ij}$ z měření ventilace si v rámci nejistot neodporují. To samé platí pro OAR, ale pouze pro varianty s TMH plynem v první zóně. To značí velikou nepřesnost měření odparů plynu TCE nebo vyhodnocování odezev TD detektorů na tento plyn.

\subsection{Závěr}
Byly určeny průměrné přísuny radonu do sklepa, přízemí a prvního patra objektu pomocí skriptu pro vyhodnocení rovnovážného měření za použití několika kombinací naměřených dat:
\begin{itemize}
    \item byly použity průměrné OAR z TERA sond a z CANARY měřáků, viz tabulky~\ref{tab:skala75_prisunyRovnovazne} a \ref{tab:skala75_prisunyRovnovazneCANARY}, to nám dává dvě varianty vypočítaných $Q_i$;
    \item dále bylo použito osm kombinací odezev TD detektorů a odparů z tří indikačních plynů pro určení osmi variant $k_{ij}$, tj. z toho máme osm variant $Q_i$.
\end{itemize}
Celkově jsme tedy získaly 16 variant přísunů radonu. Porovnáním $Q_i$ a $Q_{zdroj}$ byla vybrána nejpřesnější varianta. Tou se stala ta, která byla určena za použití OAR naměřených CANARY měřáky a za použití tracerů (TMH, MCH, PCE). Všechny $Q_i$ určené z dat CANARY měřáků a z kombinací tracerů s TMH plynem v první zóně odpovídají $Q_{zdroj}$.

Pro ověření funkčnosti skriptu pro dynamické měření byly vypočteny i časové závislosti přísunů radonu $Q_i(t)$ pro OAR z TERA sond. Bohužel $Q_i(t)$ není možné kvůli známosti pouze průměrných hodnot $k_{ij}$ považovat za dostatečně přesné. Proto byly alespoň $Q_i(t)$ zprůměrovány a tyto průměry srovnány s příslušnými $Q_i$. $\overline{Q_i}(t)$ a $Q_i$ jsou v podstatě totožné a to potvrzuje správnost skriptu pro vyhodnocování dynamického měření. Správnost skriptu pro vyhodnocování rovnovážného měření byla ověřena v kapitole~\ref{navesti:rovnovaznaMereni}.

\section{Objekt Hálková 980, Humpolec}
Jedná se o byt s ložnicí, obývacím pokojem, kuchyní s WC, špajzem a předsíní. Za zóny byly brány obývací pokoj, ložnice, koupelna s WC a kuchyň. Špajz a předsín nebyly zahrnuty do žádné ze zón.

Bylo použito dvacet vyvíječů pěti různých indikačních plynů, čtyři vyvíječe pro daný plyn. Do obývacího pokoje byly umístěny vyvíječe s MDC plynem, do ložnice vyvíječe s MCH a PCE plyny, do koupelny s TCE plynem a do kuchyně s TMH plynem. Díky umístění vyvíječů dvou plynů do ložnice lze vyhodnotit $k_{ij}$ dvěma způsoby.

TD detektory byly po dvojicích umístěny do každé ze zón kromě koupelny, kam byl umístěn pouze jeden TD detektor. Poslední TD detektor byl umístěn do špajzu, protože původně bylo zamýšleno brát jako jednu zónu koupelnu a špajz. %PORUŠENÍ PRAVIDEL OSAZOVÁNÍ MĚŘIDEL!!!!

Do každé zóny byly umístěny jedna TERA sonda a jeden CANARY měřák. Do obývacího pokoje byl umístěn radonový zdroj s označením 38 (dle tab.~\ref{tab:dynMer_zdroje}).

V příloze~\ref{navesti:priloha_halkova980} jsou uvedeny naměřené a vyhodnocené veličiny potřebné pro výpočet přísunů radonu.

V dalších oddílech této kapitoly jsou sepsány do tabulek vypočítané $Q_i$ z OAR naměřených TERA sondami a CANARY měřáky a z obou kombinací použitých tracerů. Pro srovnání jsou tam uvedeny i $Q_{zdroj}$. Dále tam lze nalézt zpětné ověření OAR a $k_{ij}$ pro kombinaci tracerů (MDC, MCH, TCE, TMH) pomocí $Q_{zdroj}$.
%\subsection{Použitá měřidla}

%\begin{itemize}
    %\setlength\itemsep{0em}
	%\item 20 vyvíječů (4x MDC, 4x MCH, 4x PCE, 4x TCE, 4x TMH)
	%\item 8 TD detektorů
	%\item 4 CANARY monitory
	%\item 4 TERA sondy
	%\item 3 TESTO měřiče teploty a vlhkosti
	%\item 1 zdroj radonu
%\end{itemize}

%\subsection{Naměřené OAR, objemy a teploty}

%\begin{table}[H]
    %\centering
    %\caption{Přiřazení číslování kompartmentů jednotlivým podlažím, objemy všech zón objektu, průměrné teploty naměřené v každém zóně TERA sondami, odhadnuté atmosférické tlaky v každém zóně a průměrné OAR naměřené TERA sondami ($OAR_T$) a CANARY měřáky ($OAR_C$). OAR jsou uvedené v \si{Bq/m^3}.}
    %\label{tab:halkova980_objemy}
    %\begin{tabular}{lll}
\toprule
podlazi & $OAR$ [\si{Bq/m^3}] & $V$ [\si{m^3}] \\
\midrule
0 &              33+/-2 &        66+/-13 \\
1 &              61+/-3 &       105+/-11 \\
2 &              79+/-2 &       153+/-15 \\
\bottomrule
\end{tabular}

%\end{table}
%\begin{figure}[H]
    %\centering
    %\includegraphics[width=1\textwidth]{halkova980/OAR_dohromady.png}
    %\caption{Vývoj OAR naměřený TERA sondami po aplikování kalibračních konstant (tab.~\ref{tab:dynMer_sondyB}).}
    %\label{fig:halkova980_OAR_dohromady}
%\end{figure}
%\begin{figure}[H]
    %\centering
    %\includegraphics[width=1\textwidth]{halkova980/OAR_CANARY.png}
    %\caption{Vývoj OAR naměřený CANARY měřáky.}
    %\label{fig:halkova980_OAR_CANARY}
%\end{figure}

%\subsection{Objemové průtoky vzduchu}

%\begin{table}[H]
    %\centering
    %\caption{Přehled použitých indikačních plynů a umístění jejich vyvíječů v objektu. V posledním sloupci jsou celkové odpary plynů ze všech jim odpovídajících vyvíječů.}
    %\label{tab:halkova980_indikacniPlyny}
    %\begin{tabular}{lrr}
\toprule
plyn & zóna&odpar [\si{mg}]\\
\midrule
 MDC & 1&   1042           \\
 MCH &2&    989            \\
 PCE &2&    317            \\
 TCE &3&    841            \\
 TMH &4&    991            \\
\bottomrule
\end{tabular}

%\end{table}
%\begin{table}[H]
    %\centering
    %\caption{Odezvy TD detektorů $R$ na všechny použité indikační plyny ve všech zónách.}
    %\label{tab:halkova980_odezvyTD}
    %\begin{tabular}{lrr}
\toprule
plyn & zóna  &    $R$\\
\midrule
MCH & 1 &    72$\pm$ 5 \\
    & 2 &  2375$\pm$99 \\
    & 3 &   203$\pm$ 9 \\
MDC & 1 &    69$\pm$ 2 \\
    & 2 &  1829$\pm$42 \\
    & 3 &   189$\pm$ 5 \\
PCE & 1 &     0$\pm$ 0 \\
    & 2 &    16$\pm$ 1 \\
    & 3 &   548$\pm$16 \\
PCH & 1 &    41$\pm$ 5 \\
    & 2 &   186$\pm$ 4 \\
    & 3 &   729$\pm$18 \\
TCE & 1 &   384$\pm$29 \\
    & 2 &   165$\pm$ 8 \\
    & 3 &   100$\pm$ 5 \\
TMH & 1 &   291$\pm$52 \\
    & 2 &   154$\pm$16 \\
    & 3 &    81$\pm$10 \\
\bottomrule
\end{tabular}

%\end{table}

%\begin{table}[H]
    %\centering
    %\caption{Objemové průtoky vzduchu mezi zónami v \si{m^3/hod} a výměna vzduchu $n$ v \si{hod^{-1}}.}
    %\label{tab:halkova980_prutoky}
    %\begin{tabular}{l
        >{\collectcell\num}r<{\endcollectcell}
        @{${}\pm{}$}
        >{\collectcell\num}r<{\endcollectcell}
        >{\collectcell\num}r<{\endcollectcell}
        @{${}\pm{}$}
        >{\collectcell\num}r<{\endcollectcell}
}
%\begin{tabular}{l>{\raggedleft\arraybackslash}p{2.5cm}>{\raggedleft\arraybackslash}p{2.5cm}}
\toprule
%{} & (MDC, PCE, TCE, TMH) & (MDC, MCH, TCE, TMH) \\
{} & \multicolumn{2}{r}{(MDC, PCE,} & \multicolumn{2}{r}{(MDC, MCH,} \\
{} & \multicolumn{2}{r}{TCE, TMH)} &   \multicolumn{2}{r}{TCE, TMH)} \\
\midrule
$k_{12}$          &    12,2&4,9 &          40,4&13,4   \\
$k_{13}$          &     4,8&8,9 &           11,9&8,2   \\
$k_{14}$          &   96,0&35,9 &          92,9&34,0   \\
$k_{21}$          &   31,3&27,5 &          40,5&14,8   \\
$k_{23}$          &    17,9&7,9 &            4,4&3,3   \\
$k_{24}$          &   13,8&24,1 &          19,7&12,7   \\
$k_{31}$          &   63,4&26,0 &          62,6&26,1   \\
$k_{32}$          &     1,9&2,8 &            6,2&8,9   \\
$k_{34}$          &   46,9&24,3 &          46,5&23,9   \\
$k_{41}$          &   76,3&37,1 &          74,6&37,7   \\
$k_{42}$          &     4,1&4,1 &          13,6&13,2   \\
$k_{43}$          &    18,0&9,7 &           20,4&9,9   \\
&\multicolumn{2}{r}{}&\multicolumn{2}{r}{}\\
$k_{1_E}$          &   72,6&18,6 &          45,3&11,8   \\
$k_{2_E}$          &  -39,9&15,3 &           12,1&4,6   \\
$k_{3_E}$          &  -27,3&12,3 &          -31,5&8,8   \\
$k_{4_E}$          &   50,9&18,7 &          41,7&13,4   \\
$k_{1_I}$          &   14,7&67,4 &          12,7&62,2   \\
$k_{2_I}$          &    5,0&41,0 &          16,6&29,1   \\
$k_{3_I}$          &   44,2&40,8 &          47,1&39,8   \\
$k_{4_I}$          &   -7,5&65,6 &          -8,8&61,4   \\
\midrule
$n$  &     0,4&0,2 &            0,4&0,1   \\   
\bottomrule
\end{tabular}

    
    
    
    
    
    
    
    
    
    
    
    
    
    
    
    
    
    
    
    
    


%\end{table}

\subsection{Přísuny radonu}

\begin{table}[H]
    \centering
    \caption{Přesně definované přísuny radonu ze zdrojů v \si{Bq/(m^3\cdot hod)}. Ve druhém sloupci je uvedeno, který zdroj byl umístěn v dané zóně.}
    \label{tab:halkova980_prisunyZdroj}
    \begin{tabular}{ll
        >{\collectcell\num}r<{\endcollectcell}
        @{${}\pm{}$}
        >{\collectcell\num}r<{\endcollectcell}
    }
        \toprule
        zóna &zdroj  & \multicolumn{2}{r}{$Q_{zdroj}$}\\
        \midrule
        1 &38& 332&64\\
        2 &NE    & 0&0   \\
        3 &NE    & 0&0   \\
        4 &NE    & 0&0   \\
        \bottomrule
    \end{tabular}
\end{table}
%\begin{table}[H]
    %\centering
    %\caption{Statistiky vypočítaných přísunů radonu $Q$ do jednotlivých podlaží.}
    %\label{tab:halkova980_prisuny}
    %\begin{tabular}{lrrr}
\toprule
{} &  $Q_0$ $\left[\si{\frac{Bq}{m^3\cdot hod}}\right]$ &  $Q_1$ $\left[\si{\frac{Bq}{m^3\cdot hod}}\right]$ &  $Q_2$ $\left[\si{\frac{Bq}{m^3\cdot hod}}\right]$ \\
\midrule
count &  478 &  478 & 478 \\
mean  & 1041 &  -22 &  19 \\
%std  &  279 &   44 &  37 \\
min   & -161 & -188 & -60 \\
25\%  &  869 &  -51 &  -6 \\
50\%  & 1044 &  -23 &  13 \\
75\%  & 1233 &    6 &  41 \\
max   & 1662 &   93 & 153 \\
\bottomrule
\end{tabular}

%\end{table}
\begin{table}[H]
    \centering
    \caption{Průměrné přísuny radonu do zón (ne podlaží!) pro všechny možné kombinace indikačních plynů za použití průměrných hodnot vývojů OAR naměřených TERA sondami (rovnovážné vyhodnocení).}
    \label{tab:halkova980_prisunyRovnovazne}
    %\begin{tabular}{lrrS[table-format=2.0(2)]}
%\toprule
%použité tracery & $Q_0$ $\left[\si{\frac{Bq}{m^3\cdot hod}}\right]$ & $Q_1$ $\left[\si{\frac{Bq}{m^3\cdot hod}}\right]$ & {$Q_2$ $\left[\si{\frac{Bq}{m^3\cdot hod}}\right]$} \\
\begin{tabular}{lrr
        >{\collectcell\num}r<{\endcollectcell}
        @{${}\pm{}$}
        >{\collectcell\num}r<{\endcollectcell}
    }
\toprule
použité tracery & $Q_0$ & $Q_1$  & \multicolumn{2}{r}{$Q_2$} \\
\midrule
(TMH, MDC, PCE) & $335\pm90$ & $236\pm42$ & 18&6 \\
(TMH, MDC, PCH) & $323\pm88$ & $231\pm42$ & 63&24 \\
(TMH, MCH, PCE) & $347\pm89$ & $197\pm36$ & 19&6 \\
(TMH, MCH, PCH) & $334\pm87$ & $192\pm35$ & 70&24 \\
&&&\multicolumn{2}{r}{}\\
(TCE, MDC, PCE) & $111\pm28$ & $249\pm41$ & 17&6 \\
(TCE, MDC, PCH) & $108\pm28$ & $243\pm41$ & 62&23 \\
(TCE, MCH, PCE) & $115\pm26$ & $208\pm35$ & 19&6 \\
(TCE, MCH, PCH) & $111\pm27$ & $203\pm35$ & 70&23 \\
\bottomrule
\end{tabular}

\end{table}

\begin{table}[H]
    \centering
    \caption{Průměrné přísuny radonu do zón pro všechny možné kombinace indikačních plynů vypočtené z průměrných hodnot vývojů OAR naměřených CANARY měřáky.}
    \label{tab:halkova980_prisunyRovnovazneCANARY}
    \begin{tabular}{lllll}
\toprule
{} & $Q_1$ $\left[\si{\frac{Bq}{m^3\cdot hod}}\right]$ & $Q_2$ $\left[\si{\frac{Bq}{m^3\cdot hod}}\right]$ & $Q_3$ $\left[\si{\frac{Bq}{m^3\cdot hod}}\right]$ & $Q_4$ $\left[\si{\frac{Bq}{m^3\cdot hod}}\right]$ \\
\midrule
(MDC, PCE, TCE, TMH) &                                         446+/-260 &                                         -25+/-105 &                                           19+/-78 &                                        -150+/-382 \\
(MDC, MCH, TCE, TMH) &                                         455+/-250 &                                         -85+/-106 &                                            6+/-76 &                                        -140+/-365 \\
\bottomrule
\end{tabular}

\end{table}

\subsection{Zpětné ověření}

\subsubsection{OAR}
\begin{table}[H]
    \centering
    \caption{Průměrné OAR ve všech zónách vypočítané pomocí rovnice~\eqref{eq:maticovy_zapis_rovnovaha} za použití průtoků vzduchu z tab.~\ref{tab:halkova980_prutoky} a přísunů radonu pocházejících od zdrojů RF 2000 (tab.~\ref{tab:halkova980_prisunyZdroj}) a průměrné OAR naměřené CANARY měřáky.}
    \label{tab:halkova980_OAR_zpetne}
    \begin{tabular}{lllll}
\toprule
{} & $Q_1$ $\left[\si{\frac{Bq}{m^3\cdot hod}}\right]$ & $Q_2$ $\left[\si{\frac{Bq}{m^3\cdot hod}}\right]$ & $Q_3$ $\left[\si{\frac{Bq}{m^3\cdot hod}}\right]$ & $Q_4$ $\left[\si{\frac{Bq}{m^3\cdot hod}}\right]$ \\
\midrule
(MDC, PCE, TCE, TMH) &200+/-167 &139+/-356 &58+/-83 &165+/-172 \\
(MDC, MCH, TCE, TMH) & 201+/-74 & 141+/-72 &59+/-46 & 166+/-74 \\
\bottomrule
\end{tabular}

\end{table}

\subsubsection{Objemové průtoky vzduchu}
\begin{table}[H]
    \centering
    \caption{V prvních řádcích těchto tabulek označených \emph{zpětně} jsou průtoky vzduchu z dané zóny do ostatních zón a infiltrace této zóny vypočítané z rovnice~\eqref{eq:maticovy_zapis_rovnovaha} za využití znalosti ostatních průtoků vzduchu pro kombinaci indikačních plynů (MDC, MCH, TCE, TMH), viz tab.~\ref{tab:halkova980_prutoky}, přísunů radonu pocházejících od zdrojů RF 2000 (tab.~\ref{tab:halkova980_prisunyZdroj}) a průměrných OAR naměřených CANARY měřáky. V druhých řádcích tabulek označených \emph{měření} jsou pro srovnání uvedené příslušné průtoky vzduchu z tab.~\ref{tab:halkova980_prutoky}. V (a) je zájmovou zónou první zóna, v (b) druhá zóna, v (c) třetí zóna a v (d) čtvrtá zóna.}
    \label{tab:halkova980_prutoky_zpetne}
    \begin{subtable}{\textwidth}
        \centering
        \caption{}
        \begin{tabular}{lrrrrr}
\toprule
{} & $k_{12}$ [\si{m^3/hod}] & $k_{13}$ [\si{m^3/hod}] & $k_{14}$ [\si{m^3/hod}] & $k_{1_E}$ [\si{m^3/hod}] & $k_{1_I}$ [\si{m^3/hod}] \\
\midrule
zpětně &               $  24\pm14$ &                 $13\pm15$ &                $ 75\pm33$ &                 $51\pm11 $&                 $19\pm62 $\\
měření &               $  40\pm13$ &                 $ 12\pm\ \,8$ &                $ 93\pm34$ &                 $45\pm12 $&                 $13\pm62 $\\
\bottomrule
\end{tabular}

    \end{subtable}
    \vspace{1em}

    \begin{subtable}{\textwidth}
        \centering
        \caption{}
        \begin{tabular}{lrrrrr}
\toprule
{} & $k_{21}$ [\si{m^3/hod}] & $k_{23}$ [\si{m^3/hod}] & $k_{24}$ [\si{m^3/hod}] & $k_{2E}$ [\si{m^3/hod}] & $k_{2I}$ [\si{m^3/hod}] \\
\midrule
zpětně &                99+/-112 &                  7+/-37 &               -20+/-102 &                 25+/-35 &                 29+/-45 \\
měření &                 40+/-15 &                   4+/-3 &                 20+/-13 &                  12+/-5 &                 17+/-29 \\
\bottomrule
\end{tabular}

    \end{subtable}
    \vspace{1em}

    \begin{subtable}{\textwidth}
        \centering
        \caption{}
        \begin{tabular}{lrrrr}
\toprule
{} & $k_{31}$ [\si{m^3/hod}] & $k_{32}$ [\si{m^3/hod}] & $k_{3E}$ [\si{m^3/hod}] & $k_{3I}$ [\si{m^3/hod}] \\
\midrule
zpětně &          -24.45+/-17.34 &            1.34+/-16.37 &           27.71+/-20.50 &           26.49+/-20.50 \\
měření &            -0.06+/-0.02 &             0.77+/-0.12 &             7.85+/-0.85 &             6.63+/-0.91 \\
\bottomrule
\end{tabular}

    \end{subtable}

    \begin{subtable}{\textwidth}
        \centering
        \caption{}
        \begin{tabular}{lrrrrr}
\toprule
{} & $k_{41}$ [\si{m^3/hod}] & $k_{42}$ [\si{m^3/hod}] & $k_{43}$ [\si{m^3/hod}] & $k_{4E}$ [\si{m^3/hod}] & $k_{4I}$ [\si{m^3/hod}] \\
\midrule
zpětně &                115+/-69 &                -10+/-26 &                 23+/-24 &                 50+/-21 &                 -0+/-63 \\
měření &                 75+/-38 &                 14+/-13 &                 20+/-10 &                 42+/-13 &                 -9+/-61 \\
\bottomrule
\end{tabular}

    \end{subtable}
\end{table}

\subsection{Diskuze}
Mnoho informací uvedených u předchozího měření v objektu Skála 75 platí i zde:
\begin{itemize}
    \item CANARY měřáky měří přesněji a spolehlivěji než TERA sondy,
    \item časové závislosti přísunů radonu $Q_i(t)$ nelze považovat za přesné.
\end{itemize}

Měření ventilace u tohoto objektu je zatíženo velikou nejistotou kvůli saturaci několika TD detektorů na některé druhy tracerů. Stalo se to v důsledku porušení pravidel osazování měřidel do objektu (podkapitola~\ref{navesti:prutoky_instalace}), jelikož do ložnice bylo umístěno moc vyvíječů, jejichž celkové emise dotyčných tracerů byly příliš veliké. Proto $k_{ij}$ vycházejí s velikou nejistotou (tab.~\ref{tab:halkova980_prutoky}) a je dost pravděpodobné, že jsou zatíženy nějakou další chybou, která není v uvedených nejistotách zahrnuta. 

V tab.~\ref{tab:halkova980_prisunyRovnovazneCANARY} jsou $Q_i$ pro data z CANARY měřáků. Kupodivu i přes uvedené nepřesné měření ventilace vychází v rámci nejistot stejně jako $Q_{zdroj}$ (tab.~\ref{tab:halkova980_prisunyZdroj}). Ovšem jejich nejistoty jsou tak obrovské, že by to při neznámosti $Q_{zdroj}$ zabraňovalo přesné kvantifikaci zdrojů radonu v objektu.

Jako přesnější byla vybrána kombinace tracerů (MDC, MCH, TCE, TMH), protože pro ní vychází méně průtoků záporně (záporné $k_{ij}$ jsou nesmyslné).

Při zpětném ověření OAR vyšly hodnoty blízké naměřeným hodnotám (tab.~\ref{tab:halkova980_OAR_zpetne}). Nominální hodnoty zpětně ověřených $k_{ij}$ vyšly značně rozdílně oproti nominálním hodnotám $k_{ij}$ z měření, avšak jejich intervaly nejistot jsou tak veliké, že se navzájem protínají (tab.~\ref{tab:halkova980_prutoky_zpetne}). 

Dynamické vyhodnocení bylo provedeno pro obě kombinace tracerů a pro OAR z TERA sond, avšak závislosti $Q_i(t)$ byly tak nepřesné (víceméně oscilovaly kolem nuly), že nejsou v této práci uvedeny.
\subsection{Závěr}
Byly určeny přísuny radonu do obývacího pokoje, ložnice, koupelny s WC a do kuchyně. Jako přesnější jsou opět brány přísuny radonu, při jejichž výpočtu byly použity OAR z CANARY měřáků. Při vyhodnocování byly uvažovány dvě kombinace tracerů, jako přesnější byla určena kombinace (MDC, MCH, TCE, TMH). Vypočítané a vybrané $Q_i$ se shoduje s $Q_{zdroj}$, avšak mají obrovskou nejistotu.

Hlavním zdrojem nejistot tohoto měření bylo umístění příliš mnoho vyvíječů do zón, což vedlo k saturaci několika TD detektorů na některé druhy tracerů.

\section{Objekt Anglická 574, Dobřichovice}
Jedná se o rodinný dům se sklepem, přízemím a prvním patrem. Při měření byla brána jednotlivá podlaží jako zóny. Do sklepa byly umístěny vyvíječe s MCH plynem, do přízemí vyvíječe s MDC plynem a do prvního patra vyvíječe s PCH plynem. Do každé zóny byly dále osazeny čtyři TD detektory. Do přízemí a do prvního patra byly dány po jednom CANARY měřáku a TERA sondě, ve sklepě byly tyto kontinuální motnitory umístěny po dvojicích. Ve sklepě byly navíc instálovány zdroje 38 a 37.

Veličiny vypočítané z měřeních jsou v příloze~\ref{navesti:priloha_anglicka574}. Přísuny radonu $Q_{zdroj}$ a $Q_i$ jsou uvedeny v dalším oddíle. Dále je v této kapitolce uvedeno zpětné ověření OAR a $k_{ij}$. Časové vývoje přísunů radonu $Q_i(t)$ z dynamického vyhodnocení OAR z TERA sond jsou také v příloze~\ref{navesti:priloha_anglicka574}.
%\subsection{Použitá měřidla}
%\begin{itemize}
    %\setlength\itemsep{0em}
	%\item 12 vyvíječů (4x MCH, 4x MDC, 4x PCH)
	%\item 12 TD detektorů
	%\item 2 blank TD detektory 
	%\item 4 CANARY monitory
	%\item 4 TERA sondy
	%\item 3 TESTO měřiče teploty a vlhkosti
	%\item 2 zdroje radonu
%\end{itemize}

%\subsection{Naměřené OAR, objemy a teploty}

%\begin{table}[H]
    %\centering
    %\caption{Objemy všech podlaží objektu, průměrné teploty naměřené v každém podlaží TERA sondami, odhadnuté atmosférické tlaky v každém podlaží, průměrné OAR naměřené TERA sondami ($OAR_T$) a CANARY měřáky ($OAR_C$) a přiřazení číslování kompartmentů jednotlivým podlažím. OAR jsou uvedené v \si{Bq/m^3}.}
    %\label{tab:anglicka574_objemy}
    %\begin{tabular}{lll}
\toprule
podlazi & $OAR$ [\si{Bq/m^3}] & $V$ [\si{m^3}] \\
\midrule
0 &              33+/-2 &        66+/-13 \\
1 &              61+/-3 &       105+/-11 \\
2 &              79+/-2 &       153+/-15 \\
\bottomrule
\end{tabular}

%\end{table}
%\begin{figure}[H]
    %\centering
    %\includegraphics[width=1\textwidth]{anglicka574/OAR_dohromady.png}
    %\caption{Vývoj OAR naměřený TERA sondami po aplikování kalibračních konstant (tab.~\ref{tab:dynMer_sondyB}).}
    %\label{fig:anglicka574_OAR_dohromady}
%\end{figure}
%\begin{figure}[H]
    %\centering
    %\includegraphics[width=1\textwidth]{anglicka574/OAR_CANARY.png}
    %\caption{Vývoj OAR naměřený CANARY měřáky.}
    %\label{fig:anglicka574_OAR_CANARY}
%\end{figure}

%\subsection{Objemové průtoky vzduchu}

%\begin{table}[H]
    %\centering
    %\caption{Přehled použitých indikačních plynů a umístění jejich vyvíječů v objektu. V posledním sloupci jsou celkové odpary plynů ze všech jim odpovídajících vyvíječů.}
    %\label{tab:anglicka574_indikacniPlyny}
    %\begin{tabular}{lrr}
\toprule
plyn & zóna&odpar [\si{mg}]\\
\midrule
 MDC & 1&   1042           \\
 MCH &2&    989            \\
 PCE &2&    317            \\
 TCE &3&    841            \\
 TMH &4&    991            \\
\bottomrule
\end{tabular}

%\end{table}
%\begin{table}[H]
    %\centering
    %\caption{Odezvy TD detektorů $R$ na všechny použité indikační plyny ve všech zónách.}
    %\label{tab:anglicka574_odezvyTD}
    %\begin{tabular}{lrr}
\toprule
plyn & zóna  &    $R$\\
\midrule
MCH & 1 &    72$\pm$ 5 \\
    & 2 &  2375$\pm$99 \\
    & 3 &   203$\pm$ 9 \\
MDC & 1 &    69$\pm$ 2 \\
    & 2 &  1829$\pm$42 \\
    & 3 &   189$\pm$ 5 \\
PCE & 1 &     0$\pm$ 0 \\
    & 2 &    16$\pm$ 1 \\
    & 3 &   548$\pm$16 \\
PCH & 1 &    41$\pm$ 5 \\
    & 2 &   186$\pm$ 4 \\
    & 3 &   729$\pm$18 \\
TCE & 1 &   384$\pm$29 \\
    & 2 &   165$\pm$ 8 \\
    & 3 &   100$\pm$ 5 \\
TMH & 1 &   291$\pm$52 \\
    & 2 &   154$\pm$16 \\
    & 3 &    81$\pm$10 \\
\bottomrule
\end{tabular}

%\end{table}

%\begin{table}[H]
    %\centering
    %\caption{Objemové průtoky vzduchu mezi zónami v \si{m^3/hod} a výměna vzduchu $n$ v \si{hod^{-1}}.}
    %\label{tab:anglicka574_prutoky}
    %\begin{tabular}{lr}
\toprule
$k_{12}$    &$2,32 \pm0,38$ \\
$k_{13}$    &$0,29 \pm0,09$ \\
$k_{21}$    &$2,96 \pm0,48$ \\
$k_{23}$    &$4,25 \pm0,65$ \\
$k_{31}$    &$0,18 \pm0,03$ \\
$k_{32}$    &$0,55 \pm0,08$ \\
&\\
$k_{1_E}$   &$22,05\pm2,53$ \\
$k_{2_E}$   &$25,24\pm2,80$ \\
$k_{3_E}$   &$30,74\pm3,07$ \\
$k_{1_I}$   &$21,52\pm2,60$ \\
$k_{2_I}$   &$29,59\pm2,94$ \\
$k_{3_I}$   &$26,93\pm3,14$ \\
\midrule
$n$ &    $0,29\pm0,04$ \\
\bottomrule
\end{tabular}

%\end{table}

\subsection{Přísuny radonu}

\begin{table}[H]
    \centering
    \caption{Přesně definované přísuny radonu ze zdrojů v \si{Bq/(m^3\cdot hod)}. Ve druhém sloupci je uvedeno, ve kterém podlaží byly zdroje umístěny.}
    \label{tab:anglicka574_prisunyZdroj}
    \begin{tabular}{ll
        >{\collectcell\num}r<{\endcollectcell}
        @{${}\pm{}$}
        >{\collectcell\num}r<{\endcollectcell}}
        \toprule
        podlaží  &zdroj& \multicolumn{2}{r}{$Q_{zdroj}$}\\
        \midrule
        0 &38 \& 37&455&90\\
        1 & NE &0&0\\
        2 & NE &0&0\\
        \bottomrule
    \end{tabular}
\end{table}

%\begin{figure}[ht]
    %\begin{subfigure}{\textwidth}
        %\centering
        %\includegraphics[width=\textwidth]{anglicka574/prisuny.png}
        %\caption{}
        %\label{fig:anglicka574_prisuny}
    %\end{subfigure}
    %\begin{subfigure}{\textwidth}
        %\centering
        %\includegraphics[width=\textwidth]{anglicka574/prisuny_zoom.png}
        %\caption{}
        %\label{fig:anglicka574_prisunyZoom}
    %\end{subfigure}
    %\caption{V (a) jsou určené časové vývoje přísunů radonu do jednotlivých podlaží. V (b) jsou přiblížené přísuny radonu do přízemí a prvního patra. Oblasti označené zesvětlenou barvou značí nejistotu přísunů radonu při faktoru pokrytí $k=1$.}
%\end{figure}
%\begin{table}[H]
    %\centering
    %\caption{Statistiky vypočítaných přísunů radonu $Q$ do jednotlivých podlaží.}
    %\label{tab:anglicka574_prisuny}
    %\begin{tabular}{lrrr}
\toprule
{} &  $Q_0$ $\left[\si{\frac{Bq}{m^3\cdot hod}}\right]$ &  $Q_1$ $\left[\si{\frac{Bq}{m^3\cdot hod}}\right]$ &  $Q_2$ $\left[\si{\frac{Bq}{m^3\cdot hod}}\right]$ \\
\midrule
count &  478 &  478 & 478 \\
mean  & 1041 &  -22 &  19 \\
%std  &  279 &   44 &  37 \\
min   & -161 & -188 & -60 \\
25\%  &  869 &  -51 &  -6 \\
50\%  & 1044 &  -23 &  13 \\
75\%  & 1233 &    6 &  41 \\
max   & 1662 &   93 & 153 \\
\bottomrule
\end{tabular}

%\end{table}
\begin{table}[H]
    \centering
    \caption{Průměrné přísuny radonu do všech podlaží z rovnovážného vyhodnocení za použití dat z TERA sond.}
    \label{tab:anglicka574_prisunyRovnovazne}
    \begin{tabular}{rrr}
        \toprule
        $Q_0$ $\left[\si{\frac{Bq}{m^3\cdot hod}}\right]$& $Q_1$ $\left[\si{\frac{Bq}{m^3\cdot hod}}\right]$ & $Q_2$ $\left[\si{\frac{Bq}{m^3\cdot hod}}\right]$\\
        \midrule
        $1042\pm233$ & $-22\pm12$ & $19\pm9$\\
        \bottomrule
    \end{tabular}
\end{table}
\begin{table}[H]
    \centering
    \caption{Průměrné přísuny radonu do všech podlaží z rovnovážného vyhodnocení za použití dat z CANARY měřáků.}
    \label{tab:anglicka574_prisunyRovnovazneCANARY}
    \begin{tabular}{rrr}
        \toprule
        $Q_0$ $\left[\si{\frac{Bq}{m^3\cdot hod}}\right]$& $Q_1$ $\left[\si{\frac{Bq}{m^3\cdot hod}}\right]$ & $Q_2$ $\left[\si{\frac{Bq}{m^3\cdot hod}}\right]$\\
        \midrule
        $1057\pm245$ & $-31\pm13$ & $21\pm7$\\
        \bottomrule
    \end{tabular}
\end{table}

\subsection{Zpětné ověření}
\subsubsection{OAR}
\begin{table}[H]
    \centering
    \caption{Průměrné OAR ve všech podlažích vypočítané pomocí rovnice~\eqref{eq:maticovy_zapis_rovnovaha} za použití průtoků vzduchu z tab.~\ref{tab:anglicka574_prutoky} a přísunů radonu pocházejících od zdrojů RF 2000 (tab.~\ref{tab:anglicka574_prisunyZdroj}) a průměrné OAR naměřené CANARY měřáky.}
    \label{tab:anglicka574_OAR_zpetne}
    \begin{tabular}{lrrr}
        \toprule
        & $a_0$ [\si{Bq/m^3}] &  $a_1$ [\si{Bq/m^3}]& $a_2$ [\si{Bq/m^3}]\\
        \midrule
       vypočítané & $1200\pm360$    & $84\pm29$ & $22\pm\ \,8$\\
       naměřené & $2770\pm196$ & $92\pm\ \,9$& $98\pm10$\\
        \bottomrule
    \end{tabular}
\end{table}
\subsubsection{Objemové průtoky vzduchu}

\begin{table}[H]
    \centering
    \caption{V prvních řádcích těchto tabulek označených \emph{zpětně} jsou průtoky vzduchu z dané zóny do ostatních zón a infiltrace této zóny vypočítané z rovnice~\eqref{eq:maticovy_zapis_rovnovaha} za využití znalosti ostatních průtoků vzduchu (viz tab.~\ref{tab:anglicka574_prutoky}), přísunů radonu pocházejících od zdrojů RF 2000 (tab.~\ref{tab:anglicka574_prisunyZdroj}) a průměrných OAR naměřených CANARY měřáky. V druhých řádcích tabulek označených \emph{měření} jsou pro srovnání uvedené příslušné průtoky vzduchu z tab.~\ref{tab:anglicka574_prutoky}. V (a) je zájmovou zónou první zóna, v (b) druhá zóna a v (c) třetí zóna.}
    \label{tab:anglicka574_prutoky_zpetne}
    \begin{subtable}{\textwidth}
        \centering
        \caption{}
        \begin{tabular}{lrrrrr}
\toprule
{} & $k_{12}$ [\si{m^3/hod}] & $k_{13}$ [\si{m^3/hod}] & $k_{14}$ [\si{m^3/hod}] & $k_{1_E}$ [\si{m^3/hod}] & $k_{1_I}$ [\si{m^3/hod}] \\
\midrule
zpětně &               $  24\pm14$ &                 $13\pm15$ &                $ 75\pm33$ &                 $51\pm11 $&                 $19\pm62 $\\
měření &               $  40\pm13$ &                 $ 12\pm\ \,8$ &                $ 93\pm34$ &                 $45\pm12 $&                 $13\pm62 $\\
\bottomrule
\end{tabular}

    \end{subtable}
    \vspace{1em}

    \begin{subtable}{\textwidth}
        \centering
        \caption{}
        \begin{tabular}{lrrrrr}
\toprule
{} & $k_{21}$ [\si{m^3/hod}] & $k_{23}$ [\si{m^3/hod}] & $k_{24}$ [\si{m^3/hod}] & $k_{2E}$ [\si{m^3/hod}] & $k_{2I}$ [\si{m^3/hod}] \\
\midrule
zpětně &                99+/-112 &                  7+/-37 &               -20+/-102 &                 25+/-35 &                 29+/-45 \\
měření &                 40+/-15 &                   4+/-3 &                 20+/-13 &                  12+/-5 &                 17+/-29 \\
\bottomrule
\end{tabular}

    \end{subtable}
    \vspace{1em}

    \begin{subtable}{\textwidth}
        \centering
        \caption{}
        \begin{tabular}{lrrrr}
\toprule
{} & $k_{31}$ [\si{m^3/hod}] & $k_{32}$ [\si{m^3/hod}] & $k_{3E}$ [\si{m^3/hod}] & $k_{3I}$ [\si{m^3/hod}] \\
\midrule
zpětně &          -24.45+/-17.34 &            1.34+/-16.37 &           27.71+/-20.50 &           26.49+/-20.50 \\
měření &            -0.06+/-0.02 &             0.77+/-0.12 &             7.85+/-0.85 &             6.63+/-0.91 \\
\bottomrule
\end{tabular}

    \end{subtable}
\end{table}

\subsection{Diskuze}
I u tohoto měření došlo k saturaci několika TD detektorů. Stalo se díky nesprávnému odhadu intenzity větrání, v tomto objektu se totiž větrá velmi málo. Kupodivu však nemají vypočítané $k_{ij}$ (tab~\ref{tab:anglicka574_prutoky}) tak velké nejistoty jako $k_{ij}$ u předchozího objektu. 

Při tomto měření přestala fungovat TERA sonda s označením 10, avšak naštěstí byla umístěna ve sklepě, kde byla ještě jedna TERA sonda, a tudíž tento výpadek další vyhodnocování neovlivnil. 

$Q_0$ (tab.~\ref{tab:anglicka574_prisunyRovnovazneCANARY}) vychází mnohem větší než $Q_{zdroj}$ (tab.~\ref{tab:anglicka574_prisunyZdroj}). Pravděpodobně to je v důsledku přirozeného přísunu radonu od geologického podloží. Není známo, zdali byla koncentrace radonu v tomto objektu někdy v minulosti proměřována, každopádně nejsou k dispozici žádné hodnoty, a proto v současné době probíhá ve sklepě a v přízemí kontinuální měření radonu. Z výsledků bude hypotéza o přísunu radonu z podloží ověřena.

V tab.~\ref{tab:anglicka574_OAR_zpetne} jsou uvedeny OAR, které by měly být naměřeny, pokud by platilo $Q_0=Q_{zdroj}$ a OAR, které bylo skutečně naměřeno CANARY měřáky. Je vidět, že rozdíl je veliký.

V tabulkách ~\ref{tab:anglicka574_prutoky_zpetne} jsou $k_{ij}$ vždy pro příslušnou zónu $i$, které byly vypočítány z naměřených OAR, $Q_{zdroj}$ a hodnot $k_{ij}$ pro ostatní zóny. Vzhledem k rozdílnosti $Q_0$ a $Q_{zdroj}$ vycházejí naprosto rozdílně od naměřených $k_{ij}$ a vlastně ani nemělo smysl počítat je. 

\subsection{Závěr}
Byly určeny průměrné přísuny radonu do jednotlivých podlaží objektu $Q_i$. Pro přízemí a první zónu vyšly v rámci nejistot shodně s příslušnými $Q_{zdroj}$, ale pro sklep je vypočítaný přísun radonu velmi odlišný od $Q_{zdroj}$. Pravděpodobně to je dáno přirozeným přísunem radonu do sklepa z geologického podloží. V současné době probíhá měření OAR za účelem ověření této úvahy.

\section{Shrnutí}
Zde jsou souhrnně uvedeny všechny vypočítané průměrné přísuny radonu, resp. přísuny radonu od zdrojů RF 2000 do všech zón všech objektů (tab.~\ref{tab:dynMer_shrnuti}, resp. tab~\ref{tab:dynMer_shrnuti_zdroj}).
\begin{table}[ht]
    \centering
    \caption{Průměrné přísuny radonu $Q_i$ do všech zón (ne podlaží!) všech objektů vypočítané z OAR naměřených CANARY měřáky. V případě objektu Skála 75, resp. Hálková 980 jsou uvedeny $Q_i$ pro kombinaci tracerů (TMH, MCH, PCE), resp. (MDC, MCH, TCE, TMH). Tyto kombinace jsou považovány za nejpřesnější.}
    \label{tab:dynMer_shrnuti}
    \begin{tabular}{l
        >{\collectcell\num}r<{\endcollectcell}
        @{${}\pm{}$}
        >{\collectcell\num}r<{\endcollectcell}
        >{\collectcell\num}r<{\endcollectcell}
        @{${}\pm{}$}
        >{\collectcell\num}r<{\endcollectcell}
        >{\collectcell\num}r<{\endcollectcell}
        @{${}\pm{}$}
        >{\collectcell\num}r<{\endcollectcell}
        >{\collectcell\num}r<{\endcollectcell}
        @{${}\pm{}$}
        >{\collectcell\num}r<{\endcollectcell}
    }
        \toprule
        {}& \multicolumn{2}{r}{$Q_1$} & \multicolumn{2}{r}{$Q_2$} & \multicolumn{2}{r}{$Q_3$} & \multicolumn{2}{r}{$Q_4$} \\
        \midrule

Skála 75 &   301&78 & 115&24 &   9&3 &  \multicolumn{2}{r}{}\\
Hálková 980 & 445&241 & -86&104 & 38&84 & -152&351 \\
Anglická 574 & 1057&245 & -31&13 & 21&7 &\multicolumn{2}{r}{}\\
\bottomrule
    \end{tabular}
\end{table}

\begin{table}[ht]
    \centering
    \caption{Přísuny radonu ze zdrojů RF 2000 do všech zón všech proměřených objektů.}
    \label{tab:dynMer_shrnuti_zdroj}
    \begin{tabular}{l
        >{\collectcell\num}r<{\endcollectcell}
        @{${}\pm{}$}
        >{\collectcell\num}r<{\endcollectcell}
        >{\collectcell\num}r<{\endcollectcell}
        @{${}\pm{}$}
        >{\collectcell\num}r<{\endcollectcell}
        >{\collectcell\num}r<{\endcollectcell}
        @{${}\pm{}$}
        >{\collectcell\num}r<{\endcollectcell}
        >{\collectcell\num}r<{\endcollectcell}
        @{${}\pm{}$}
        >{\collectcell\num}r<{\endcollectcell}
    }
        \toprule
        {}& \multicolumn{2}{r}{$Q_1$} & \multicolumn{2}{r}{$Q_2$} & \multicolumn{2}{r}{$Q_3$} & \multicolumn{2}{r}{$Q_4$} \\
        \midrule

Skála 75 & 400& 51& 114& 13& 0& 0&  \multicolumn{2}{r}{}\\
Hálková 980 & 332& 64& 0&0 & 0& 0& 0& 0\\
Anglická 574 & 455& 90& 0& 0& 0& 0&\multicolumn{2}{r}{}\\
\bottomrule
    \end{tabular}
\end{table}
