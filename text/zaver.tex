\chapter{Závěr}
V této práci byly nejprve uvedeny metody pro měření OAR, poté bylo probráno měření intenzity větrání pomocí techniky indikačních plynů. Ve třetí kapitole byl odvozen odvozen model pro výpočet přísunů radonu do kompartmentů a v praktické části byl ověřen na naměřených datech. Nejprve ve čtvrté kapitole bylo dokázáno, že model pro vyhodnocování průměrných hodnot $Q_i$ dává správné výsledky. V páté kapitole pak bylo ukázáno, že časové vývoje $Q_i(t)$ nelze jen ze znalosti časových vývojů OAR v kompartmentech smysluplně určit. Aby to tak bylo, pak bychom potřebovali určit i časové vývoje objemových průtoků mezi zónami, což není v rámci uvedené metody měření ventilace objektu možné.

Dále se ukázalo, že TERA sondy neměří dostatečně přesně a i jejich spolehlivost je chabá (při jednom měření jedna sonda přestala fungovat, při jiném měření byly potíže s vyčítáním dat). Naopak CANARY měřáky se ukázaly jako spolehlivé kontinuální monitory.

V rámci této práce byla provedena tři měření ve třech různých objektech (viz již zmiňovaná pátá kapitola). Do některých zón těchto objektů byly dány umělé zdroje radonu a předpokládaje nulové přirozené přísuny radonu jsme takto mohli ověřit vypočítané $Q_i$. Ve prvních dvou objektech se $Q_i$ shodují s $Q_{zdroj}$, i když u druhého byly určeny s velikou nejistotou díky technické chybě při osazování měřidel při měření ventilace objektu. Ve třetím objektu se $Q_i$ a $Q_{zdroj}$ neshodují. Ze zpětně dopočítaného OAR z $Q_{zdroj}$ se ukázalo, že je dáno příliš vysokou hodnotou OAR ve sklepě a tedy příčinou neshody je pravděpodobně nenulový přirozený přísun radonu do sklepa. V současné době probíhá kontinuální měření OAR ve sklepě tohoto objektu pro ověření
této úvahy. U prvního objektu bylo použito dvojnásobné množství indikačních plynů než bylo potřeba. Díky tomu jsme mohli při výpočtu $k_{ij}$ uvažovat osm kombinací tracerů a tím jsme dostali osm variant přísunů radonu. Ukázalo se, že $k_{ij}$ se velmi liší v závislosti na použitých tracerech, což se odrazilo i v rozdílnosti různých variant $Q_i$. Rozhodně by bylo zajímavé provést další měření pro zjištění rozdílnosti chování různých tracerů. Díky znalosti $Q_{zdroj}$ a zpětnému dopočítání OAR a $k_{ij}$ z $Q_{zdroj}$ byla určena kombinace tracerů dávající nejpřesnější výsledky.

Provedená měření dokazují, že odvozený model funguje a lze ho používat v dalších měřeních. Uplatnění najde hlavně ve hledání zdrojů radonu v budovách. V dalším výzkumu by bylo zajímavé sledovat vlivy různých vnějších faktorů (např. roční období, vítr, déšť, teplota) na hodnoty $k_{ij}$, resp. $Q_i$. Také by bylo vhodné provést další zkušební měření pro lepší zvládnutí techniky měření s menším děláním chyb.

%\section{Poznamky}
%-vlivy na vysledky: rocni obdobi, ve kterem mereni probiha atd; teplota, tlak, vitr, dest (v pripade uvazovani koncentrace vnejsiho prostredi) atd.\\
%-je prisun radonu vetsi v zime nebo v lete?\\
%-Využití, použití
%\section{Dotazy}
%\begin{enumerate}
	%\item Proč se se zvyšující se venkovní OAR přísuny radonu zmenšují? (viz rovnovážná měření)
	%\item Část indikačních plynů se nasorbuje do materiálů k tomu vhodných (koberec), platí to i pro radon? (myslím, že Lenk říkal, že ano, ale jistý si tím nejsem.) U indikačních plynů se takto nasorbuje klidně až 20 \%.
	%\item Počítají se nejistoty u Bulharských konstant pro TERA sondy?
    %\item Jaký typ detektoru jsou TERA sondy a CANARY měřáky? TERA sondy jsou polovodiče, CANARY zcela určitě taky.
    %\item Kde lze dohledat $U_R$ a $M_w$? Je molekulová hmotnost ($M_w$) to samé jako molární hmotnost? Lze použít $U_R$ udávané v metodice?
%\section{TO DO}
%\begin{enumerate}
    %\item radon: přidat fotky kontinuálů
%\end{enumerate}
