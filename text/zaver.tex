\chapter{Závěr}
-vlivy na vysledky: rocni obdobi, ve kterem mereni probiha atd; teplota, tlak, vitr, dest (v pripade uvazovani koncentrace vnejsiho prostredi) atd.\\
-je prisun radonu vetsi v zime nebo v lete?\\
-Využití, použití
\section{Dotazy}
\begin{enumerate}
	\item Proč se se zvyšující se venkovní OAR přísuny radonu zmenšují? (viz rovnovážná měření)
	\item Část indikačních plynů se nasorbuje do materiálů k tomu vhodných (koberec), platí to i pro radon? (myslím, že Lenk říkal, že ano, ale jistý si tím nejsem.) U indikačních plynů se takto nasorbuje klidně až 20 \%.
	\item Počítají se nejistoty u Bulharských konstant pro TERA sondy?
    \item Jaký typ detektoru jsou TERA sondy a CANARY měřáky? TERA sondy jsou polovodiče, CANARY zcela určitě taky.
    \item Kde lze dohledat $U_R$ a $M_w$? Je molekulová hmotnost ($M_w$) to samé jako molární hmotnost? Lze použít $U_R$ udávané v metodice?
\section{TO DO}
\begin{enumerate}
    \item radon: přidat fotky kontinuálů
\end{enumerate}
\end{enumerate}
